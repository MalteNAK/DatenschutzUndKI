\documentclass[11pt]{article}
\usepackage[table]{xcolor}
\usepackage[ngerman]{babel} % Für Silbentrennung
\usepackage[utf8]{inputenc} % Um Umlaute und Sonderzeichen nutzen zu können
\usepackage[a4paper,
lmargin={2.5cm}, % 2.5cm links
rmargin={2.5cm}, % 2.5cm rechts
tmargin={2.5cm},
bmargin={2.75cm}]{geometry} % Für das Layout
\usepackage{setspace} % Für den Zeilenabstand
\setstretch{1.25}
\usepackage{graphicx} % Um Bilder einzufügen
\usepackage{wrapfig} % Um Bilder im Fließtext einzubinden
\usepackage{csquotes} % Um Text in Anführungsstriche zu setzen
\usepackage{mathptmx} % Ähnlich zu Times New Roman
\usepackage{fancyhdr} % Um den Header einzufügen
\usepackage{pdfpages} % Um die Titelseite einbinden
\usepackage{hyperref} % Um Verlinkungen einfügen
\usepackage{acronym} % Um Akronyme zu definieren
\usepackage{textcomp} % Um spezielle Zeichen verwenden zu können (z.B. °, µ)
\usepackage{subcaption} % Um mehrere Bilder zusammen einzustellen und Links zu Abbildungen zu korrigieren
\usepackage{tabularx}
\usepackage{booktabs}
\usepackage{caption}
\usepackage{enumitem}
\usepackage{float}
\usepackage{csquotes}

\usepackage[
		backend=bibtex,
		style=authoryear,
		maxcitenames=2,
		mincitenames=1,
		maxbibnames=99
]{biblatex} % Harvard-Stil mit "et al."-Kürzung
\addbibresource{Quellenverzeichnis.bib}
\DefineBibliographyStrings{ngerman}{andothers={et al.}}

% Zitierungen kursiv innerhalb runder Klammern setzen
\renewcommand*{\mkbibparens}[1]{\mkbibemph{(#1)}}
\let\cite\parencite

\setlength{\parindent}{0pt} % Verhindert das Einrücken des ersten Satzes jedes Absatzes

\newcommand{\note}[1]{%
  {\color{red} #1}%
}


\pagestyle{fancy}
\fancyhf{} % Leert den Kopf- und Fußbereich
\RequirePackage{soul}
% Kopfzeilenbreite auf den Textbereich anpassen
\setlength{\headheight}{42.0pt} % Setzt die Kopfzeilenhöhe
\setlength{\headsep}{25pt} % Setzt den Abstand zwischen der Linie und dem Textkörper
\renewcommand{\headrulewidth}{2pt} % Stellt die Linienbreite auf 2pt
\setlength{\headwidth}{\dimexpr\textwidth} % Setzt die Linie so weit wie der Textbereich geht

% Kapitelname links und Logo rechts
\fancyhead[L]{\nouppercase{\leftmark}} % Zeigt das Kapitel links an
\fancyhead[R]{\includegraphics[width=3cm]{Resources/Nordakademie_Logo.png}} % Zeigt das Logo rechts an

\cfoot{\thepage} % Fußzeile mit Seitenzahl

\newcounter{romanBeginningEnd} % Um Römische Seitenzahl zu speichern

% Sektionen referenzen anpassen
\addto\extrasngerman{%
	\renewcommand{\sectionautorefname}{siehe}%
	\let\subsectionautorefname\sectionautorefname%
	\let\subsubsectionautorefname\sectionautorefname%
}

\begin{document}
    \pagenumbering{gobble}
    \begin{titlepage}
    \begin{center}

        \includegraphics[width=0.9\textwidth]{Resources/Nordakademie_Logo.png}
        
        \vspace{1cm}
        
        \huge
        \textbf{Thematiges Thema}
             
        \vspace{1.5cm}

        \Large
        vorgelegt von:\\
        \vspace{0.5cm}
        \textbf{Bli}\\
        Matr. Nr.: 1 \\
        \Large
        und\\
        \textbf{Bla}\\
        Matr. Nr.: 2\\
        \Large
        und\\
        \textbf{Blub}\\
        Matr. Nr.: 3
        
            
        \vspace{3.5cm}

    \end{center}
        \Large
        \textbf{Dozent:} Dozentiger Dozent\\
        \textbf{Modul:} Doofes Modul
            
\end{titlepage}

	\setcounter{page}{1} % Römische Seitenzahlen
	\pagenumbering{Roman}
	
	\tableofcontents % Inhaltsverzeichnis
	\newpage
	
	\setcounter{secnumdepth}{0} % Keine Numerierung von Überschriften
	\clearpage
	\phantomsection

	\section{Abkürzungsverzeichnis}
	% Abkürzungen


    \begin{acronym}[DSGVO] % Für Formatierung längste Abkürzung eintragen
        \acro{GPAI}{General-Purpose AI Models}
		\acro{ML}{Machine Learning}
        \acro{UrhG}{Urheberrechtsgesetz}
        \acro{XAI}{Explainable Artificial Intelligence}
		\acro{LLM}{Large Language Model}
		\acro{RLHF}{Reinforcement Learning from Human Feedback}
    \end{acronym}


    
	\newpage
	
	\setcounter{secnumdepth}{3} % Überschriften wieder numerieren
	
	\setcounter{romanBeginningEnd}{\the\value{page}} % Speichern der römischen Seitenzahl
	\setcounter{page}{1} % Mit Arabischen Seitenzahlen wieder bei 1 anfangen
	\pagenumbering{arabic}
	
	% ###############################################################
	% Hier Inhalt einfügen
	\section{Einleitung}
\subsection{Hintergrund}
Die rasante Entwicklung generativer KI-Systeme wie großer Sprach- und Bildmodelle stellt zentrale Begriffe moderner Kultur- und Geisteswissenschaften infrage. Insbesondere das Konzept der Autorschaft gerät unter Druck, wenn Texte, Bilder oder Musik scheinbar autonom von technischen Systemen erzeugt werden. Werke, die bislang als Ausdruck individueller Kreativität galten, entstehen zunehmend in komplexen Mensch-Maschine-Konstellationen, deren Beiträge sich nicht eindeutig trennen lassen. Diese Entwicklung wirft grundlegende philosophische, technische und rechtliche Fragen auf: Wer ist Autor eines KI-generierten Werkes? Welche Rolle spielen Intentionalität, Originalität und Verantwortung in einem Prozess, der auf statistischer Mustererkennung beruht? Und wie lassen sich bestehende Konzepte geistigen Eigentums und kreativer Leistung unter diesen Bedingungen sinnvoll anwenden oder neu denken? Der Diskurs um KI-gestützte Kreativität ist dabei nicht nur theoretischer Natur, sondern auch von hoher gesellschaftlicher Relevanz, etwa im Hinblick auf Urheberrecht, kulturelle Produktion, akademische Praxis und kreative Berufe.
\subsection{Zielstellung}
Die vorliegende Arbeit nähert sich diesen Fragen aus verschiedenen Perspektiven. Ausgangspunkt ist ein philosophisches Verständnis von Autorschaft, das historisch gewachsen ist und insbesondere in der Ästhetik der Aufklärung und der Moderne ausgearbeitet wurde. Dabei werden sowohl individualistische Konzepte, etwa bei Immanuel Kant, als auch kritischere Perspektiven, welche Autorschaft als kollektive Kreativität ansehen, bei Autoren wie Roland Barthes und Michel Foucault berücksichtigt. Darauf aufbauend wird untersucht, wie sich Autorschaft im Kontext algorithmischer Kreativität technisch rekonfigurieren lässt und warum KI-Systeme die Zuschreibung von Urheberschaft strukturell erschweren. Folglich werden die ethischen Implikationen dieser Veränderungen diskutiert, insbesondere im Hinblick auf Verantwortung, Anerkennung und den normativen Status geistiger Leistungen. Hierzu wird sich auf aktuelle Richtlinien und Gesetzestexte, wie beispielsweise dem AI-Act, bezogen. Schließlich werden diese Vorschriften, aus einer moralischen und technischen Sicht, basierend auf den erarbeiteten Aspekten, bewertet. 
	\section{Grundlagen: Autorschaft und Künstliche Intelligenz}
\subsection{Historischer und philosophischer Begriff der Autorschaft}
Heutzutage, sowie seit der Neuzeit und insbesondere im Zuge der Aufklärung und der, im späten 18. Jahrhundert aufkommenden romantischen Genieästhetik, beherrscht die Rolle des Autors den Diskurs von fast jedem erdachten Werk (vgl. Barthes, \textit{Der Tod des Autors}). Im Folgenden soll das Autorschaftsverständnis untersucht werden, indem auf die Produktion eines kreativen Werks eingegangen wird. Zum Einen durch ein schöpferes Individuum oder durch die Kreation in kollektiver Form. Zudem werden die damit verbundenen Konzepte von Originalität, Inspiration und geistiges Eigentum behandelt, um den Einfluss von KI auf das Thema Autorschaft zu diskutieren.
\subsubsection{Das schöpferische Individuum}
Hier werd ich mich größtenteils auf Kant beziehen.\\
Kernaussage:\\
schöpferes Wesen (in unserem Fall der Autor), das in der Lage ist, moralische Entscheidungen zu treffen und nach moralischen Prinzipien zu handeln.\\
Autorschaft 
\subsubsection{Kollektive Kreativität}
Hier werde ich mich auf Michel Foucaults Werke "Was ist ein Autor" und "Autorfunktion", sowie Barthes "Der Tod des Autors." beziehen.\\
Kernaussage:\\
Relevanz des Autors\\
Der Autor ist nicht die letzte Quelle der Bedeutung eines Textes, diese entsteht eher im Lesen und im kulturellen Kontext.\\
Texte als Gewebe aus Zitaten, nicht Ausdruck des inneren Selbst.\\
Autor als Funktion des Diskurses und nicht als natürliche Person.\\
\textbf{Autorschaft erfüllt gesellschaftliche Aufgaben:\\
Zuschreibung\\
Verantwortungszuteilung\\
Kanonisierung\\
Kontrolle von Diskursen\\
--> Autorschaft entsteht dort, wo Zurechenbarkeit notwendig ist.\\} 
KI hat keinen Selbstbezug, keine Fähigkeit Gründe zu haben, keine Teilnahme an Diskursen als Subjekt oder Person.\\
\subsubsection{Originalität, Inspiration und geistiges Eigentum}
Weiß ich noch nicht. :/
	\section{Technische Perspektive: Autorschaft in der algorithmischen Kreativität}

Nachdem im vorherigen Kapitel die philosophisch-historischen Grundlagen der Autorschaft sowie die technischen Prinzipien generativer KI dargelegt wurden, widmet sich dieses Kapitel der Frage, wie sich Autorschaft aus technischer Perspektive im Zeitalter algorithmischer Kreativität neu denken lässt. Während Kapitel 2 erörtert hat, warum KI den traditionellen Autorschaftsbegriff herausfordert – insbesondere durch die Grenzen maschineller Intentionalität und die Reproduktionslogik neuronaler Netze –, konzentriert sich die folgende Analyse auf die praktische Werkgenese in KI-Systemen. Dabei wird untersucht, wie aus einem Prompt ein kreatives Werk entsteht, welche Akteure und Prozesse am kreativen Akt beteiligt sind, und wie die Black-Box-Problematik die Zuschreibung von Autorschaft erschwert. Zudem wird beleuchtet, wie die Mensch-Maschine-Kooperation die Rolle des Autors fundamental verändert – vom Schöpfer zum Kurator algorithmischer Prozesse.

\subsection{Der kreative Prozess in KI-Systemen}

Die Generierung eines KI-basierten Werkes folgt einem grundlegenden Ablauf, der sich vereinfacht als Dreischritt darstellen lässt: \textit{Prompt} → \textit{Modell} → \textit{Output}. Dieser Prozess unterscheidet sich fundamental von der traditionellen Werkschöpfung, da er eine Vielzahl technischer Komponenten einbezieht, die jeweils einen Beitrag zum Endergebnis leisten.

\subsubsection{Von der Eingabe zur Ausgabe: Die technische Pipeline}

Der Prozess beginnt mit dem \textit{Prompt}, einer textuellen, visuellen oder multimodalen Eingabe durch einen menschlichen Nutzer. Dieser Prompt fungiert als Initialschritt und definiert die Richtung des zu generierenden Inhalts. Bei Textgeneratoren wie GPT-5 oder Claude kann dies eine einfache Anweisung sein (\textit{„Schreibe ein Gedicht über den Winter''}), während bei Bildgeneratoren wie DALL-E oder Midjourney detaillierte Beschreibungen gewünschter Stilelemente, Kompositionen und Atmosphären verwendet werden.

Die eigentliche Transformation erfolgt im \textit{Modell}, einem komplexen neuronalen Netzwerk, das auf Basis von Millionen bzw. Milliarden Parametern arbeitet. Diese Parameter wurden während des Trainingsprozesses aus großen Datensätzen gelernt und repräsentieren statistische Muster, die das Modell befähigen, auf neue Eingaben mit plausiblen Ausgaben zu reagieren. Bei generativen Modellen wie Transformern oder Diffusionsmodellen geschieht dies durch iterative Prozesse, bei denen das Modell schrittweise eine Ausgabe konstruiert – etwa indem es Wort für Wort einen Text generiert oder Rauschen in ein kohärentes Bild umwandelt.

Der \textit{Output} ist schließlich das generierte Werk: ein Text, ein Bild, ein Musikstück oder eine andere kreative Leistung. Dieser Output ist jedoch nicht deterministisch – selbst bei identischen Prompts können verschiedene Ausgaben entstehen, da viele Modelle mit Zufallselementen (Sampling-Strategien) arbeiten, um Diversität und Kreativität zu fördern.

\subsubsection{Technische Urheberschaft: Daten, Algorithmen und Modelle als ``Mit-Autoren''}

Die Frage, wer als Urheber eines KI-generierten Werkes anzusehen ist, wird technisch komplexer, wenn man die verschiedenen Komponenten der Werkschöpfung betrachtet. Mindestens vier Akteure oder Entitäten sind am Entstehungsprozess beteiligt:

\begin{enumerate}
    
    \item \textbf{Die Trainingsdaten:} Jedes KI-Modell basiert auf einem Datensatz, der oft aus urheberrechtlich geschützten Werken besteht. Ein Bildgenerator, der auf Millionen von Kunstwerken trainiert wurde, reproduziert implizit Stilmerkmale und Kompositionsprinzipien dieser Werke. Die ursprünglichen Künstler könnten argumentieren, dass ihre kreativen Leistungen als Grundlage für die KI-Generierung dienen und sie somit indirekt „Mit-Autoren'' sind.
    
    \item \textbf{Die Entwickler des Algorithmus:} Die Architektur eines neuronalen Netzwerks – etwa die Entscheidung für ein Transformer-Modell oder ein Diffusionsmodell – bestimmt maßgeblich, wie das Modell lernt und generiert. Die Entwickler, die diese Algorithmen entwerfen, legen technische Rahmenbedingungen fest, die den kreativen Prozess strukturieren.
    
    \item \textbf{Das trainierte Modell:} Das Modell selbst, als Ergebnis des Trainings, repräsentiert eine spezifische Konfiguration von Parametern, die aus den Daten gelernt wurden. Es ist die „ausführende Instanz'', die den Prompt in einen Output transformiert. Ob das Modell als eigenständiger Akteur betrachtet werden kann, bleibt umstritten – es agiert deterministisch (oder pseudo-stochastisch), besitzt aber kein Bewusstsein oder intentionale Handlungsfähigkeit.
    
    \item \textbf{Der Nutzer (Prompt-Geber):} Der Mensch, der den Prompt formuliert und möglicherweise iterativ verfeinert, hat ebenfalls einen wesentlichen Einfluss auf das Ergebnis. Je detaillierter und kreativer der Prompt, desto stärker kann argumentiert werden, dass der Nutzer eine gestalterische Rolle übernimmt.
\end{enumerate}

Diese Vielschichtigkeit führt zu einer \textit{verteilten Urheberschaft}, bei der es schwerfällt, eine einzelne Person oder Entität als alleinigen Autor zu identifizieren. Technisch betrachtet ist das generierte Werk das Resultat einer Kette von Entscheidungen und Prozessen, die von menschlichen und nicht-menschlichen Akteuren gemeinsam getragen werden.

\subsection{Transparenz und Nachvollziehbarkeit}

Ein zentrales Problem bei der Bewertung von KI-generierter Kreativität ist die mangelnde Transparenz vieler Modelle. Die sogenannte Black-Box-Problematik beschreibt das Phänomen, dass selbst die Entwickler eines neuronalen Netzwerks nicht immer nachvollziehen können, warum ein Modell eine bestimmte Ausgabe produziert hat.

\subsubsection{Black-Box-Problematik und Explainable AI}

Neuronale Netzwerke bestehen aus Schichten von Neuronen, die durch Gewichte miteinander verbunden sind. Diese Gewichte werden während des Trainings so angepasst, dass das Modell auf die Trainingsdaten möglichst gut reagiert. Der resultierende Parameterraum ist jedoch so hochdimensional, dass eine Interpretation der internen Entscheidungsprozesse äußerst schwierig ist. Ein Bildgenerator mag ein beeindruckendes Porträt erzeugen, aber welche spezifischen Merkmale aus welchen Trainingsdaten in welcher Kombination zu diesem Ergebnis geführt haben, bleibt oft verborgen.

Diese Intransparenz hat weitreichende Konsequenzen für die Zuschreibung von Autorschaft. Wenn nicht nachvollziehbar ist, wie ein Werk entstanden ist, wird es schwierig zu bestimmen, welchen Anteil die einzelnen Komponenten (Daten, Algorithmus, Prompt) am kreativen Prozess hatten. Zudem erschwert es die rechtliche und ethische Bewertung: Hat das Modell unzulässig urheberrechtlich geschützte Inhalte reproduziert? Wurde der Prompt „kreativ genug'' formuliert, um eine eigene Leistung darzustellen?

Als Reaktion auf diese Problematik hat sich das Forschungsfeld der \textit{Explainable AI (XAI)} entwickelt. XAI-Methoden versuchen, die Entscheidungsprozesse von KI-Systemen transparenter zu machen, etwa durch Visualisierungen (z. B. Attention-Maps, die zeigen, welche Bereiche eines Inputs das Modell fokussiert hat) oder durch vereinfachte Ersatzmodelle, die das Verhalten komplexer Netzwerke approximieren. Im Kontext der Autorschaft könnten solche Methoden helfen, den kreativen Beitrag einzelner Komponenten besser zu quantifizieren und somit eine gerechtere Zuschreibung zu ermöglichen.

\subsubsection{Bedeutung für die Zuschreibung von Autorenschaft}

Transparenz ist nicht nur eine technische, sondern auch eine normative Anforderung. Ohne Nachvollziehbarkeit des Entstehungsprozesses wird es unmöglich, die im vorherigen Kapitel diskutierten philosophischen Konzepte von Autorschaft – Intentionalität, Originalität, Schöpfertum – auf KI-generierte Werke anzuwenden. Wenn unklar bleibt, ob ein Werk tatsächlich „neu'' ist oder lediglich eine Rekombination existierender Muster darstellt, verliert der Begriff der Originalität an Bedeutung.

Gleichzeitig stellt die Black-Box-Problematik die Vorstellung eines klar identifizierbaren Autors infrage. In der traditionellen Autorschaft ließ sich der Schöpfungsprozess – zumindest theoretisch – auf die Intentionen und Handlungen eines Individuums zurückführen. Bei KI-Systemen hingegen ist die Werkgenese so komplex und verteilt, dass eine eindeutige Zuschreibung kaum möglich ist. Dies legt nahe, dass neue Konzepte von Autorschaft entwickelt werden müssen, die nicht auf einem singulären, intentionalen Schöpfer basieren, sondern die Vielzahl der beteiligten Akteure und Prozesse anerkennen.

\subsection{Mensch-Maschine-Kooperation}

Die zunehmende Verbreitung von KI-Tools in kreativen Bereichen führt zu einer grundlegenden Verschiebung der Rollen im Schöpfungsprozess. Anstatt dass der Mensch allein kreativ tätig wird, entsteht eine \textit{Co-Creation}, bei der Mensch und Maschine gemeinsam ein Werk hervorbringen. Diese Kooperation wirft neue Fragen auf: Wer ist Autor, wenn beide Seiten zum Ergebnis beitragen? Und wie verändert sich das Selbstverständnis kreativer Arbeit, wenn Algorithmen zentrale Aufgaben übernehmen?

\subsubsection{Co-Creation: Der Autor als Kurator statt Schöpfer}

In der Zusammenarbeit mit generativen KI-Systemen verschiebt sich die Rolle des Menschen vom klassischen Schöpfer hin zu einem \textit{Kurator} oder \textit{Dirigenten}. Der Nutzer gibt nicht mehr jeden Pinselstrich oder jedes Wort selbst vor, sondern formuliert Vorgaben, wählt aus verschiedenen generierten Varianten aus und verfeinert das Ergebnis iterativ. Ein Künstler, der mit Midjourney arbeitet, könnte beispielsweise Dutzende von Bildvarianten generieren lassen, aus denen er die vielversprechendsten auswählt, um sie weiter zu bearbeiten oder zu kombinieren.

Diese kuratorische Tätigkeit erfordert durchaus kreative Kompetenzen: die Fähigkeit, präzise und inspirierende Prompts zu formulieren, ein ästhetisches Urteilsvermögen, um gelungene Outputs zu erkennen, und die Fertigkeit, das Rohmaterial der KI in ein kohärentes Werk zu integrieren. Dennoch unterscheidet sich diese Tätigkeit qualitativ von der traditionellen Kunstproduktion, bei der der Künstler die volle Kontrolle über jeden Aspekt des Werkes behält.

Es entsteht eine neue Form der Autorschaft, die man als \textit{kollaborative Autorschaft} bezeichnen könnte: Der Mensch liefert die Idee und trifft ästhetische Entscheidungen, während die KI die technische Umsetzung übernimmt. Diese Aufteilung ist nicht unproblematisch, denn sie verwischt die Grenzen zwischen eigenständiger kreativer Leistung und bloßer Anwendung eines Werkzeugs. Während ein Maler mit Pinsel und Farbe arbeitet, ohne dass dies seine Urheberschaft infrage stellt, ist die KI kein passives Werkzeug – sie trifft eigenständige (wenn auch nicht intentionale) „Entscheidungen'' über die Gestaltung des Outputs.

\subsubsection{Verschiebung der Rollen im kreativen Prozess}

Die Mensch-Maschine-Kooperation verändert nicht nur die Autorschaft einzelner Werke, sondern auch das gesamte Verständnis von Kreativität. Traditionell galt Kreativität als eine genuin menschliche Fähigkeit, die auf Imagination, emotionaler Tiefe und kulturellem Verständnis beruht. KI-Systeme hingegen operieren auf der Grundlage statistischer Muster und besitzen weder Bewusstsein noch emotionale Erfahrungen. Dennoch sind sie in der Lage, Outputs zu erzeugen, die ästhetisch ansprechend, überraschend oder sogar „originell'' wirken.

Dies führt zu einer paradoxen Situation: Einerseits können KI-Systeme keine intentionalen Schöpfer sein, da ihnen die Fähigkeit zur Selbstreflexion und zur bewussten Gestaltung fehlt. Andererseits produzieren sie Werke, die von menschlichen Betrachtern als kreativ wahrgenommen werden. Diese Diskrepanz legt nahe, dass Kreativität möglicherweise nicht ausschließlich an Intentionalität gebunden ist, sondern auch als Eigenschaft eines Prozesses oder eines Ergebnisses verstanden werden kann.

Im Zuge dieser Entwicklung verschieben sich auch die Anforderungen an kreative Berufe. Designer, Autoren und Künstler müssen zunehmend lernen, mit KI-Tools umzugehen, was neue Kompetenzen erfordert: Prompt Engineering, das Verständnis von Modellfähigkeiten und -limitationen sowie die Fähigkeit, KI-generierte Inhalte kritisch zu bewerten und zu integrieren. Gleichzeitig besteht die Gefahr, dass kreative Prozesse standardisiert und homogenisiert werden, wenn viele Schaffende auf dieselben Modelle und Trainingsdaten zurückgreifen.

\subsection{Zwischenfazit: Technische Rekonzeption der Autorschaft}

Die technische Analyse des kreativen Prozesses in KI-Systemen offenbart eine fundamentale Herausforderung für das traditionelle Verständnis von Autorschaft. Die Werkgenese ist nicht mehr einer einzigen, intentional handelnden Person zuzuschreiben, sondern verteilt sich auf eine Vielzahl von Akteuren und Prozessen: die Ersteller der Trainingsdaten, die Entwickler der Algorithmen, das trainierte Modell selbst und den Nutzer, der den Prompt formuliert.

Die Black-Box-Problematik verschärft diese Situation, da sie es erschwert, den Beitrag einzelner Komponenten nachzuvollziehen und zu bewerten. Ohne Transparenz bleibt unklar, ob ein Werk tatsächlich originell ist oder lediglich eine Rekombination bekannter Muster darstellt. Dies stellt nicht nur rechtliche, sondern auch philosophische Konzepte von Autorschaft infrage, die im zweiten Kapitel dargelegt wurden – insbesondere die Vorstellung eines intentionalen, selbstreflexiven Schöpfers.

Zugleich verändert die Mensch-Maschine-Kooperation die Rolle des menschlichen Schöpfers. Anstatt eines autonomen Autors, der ein Werk von Grund auf erschafft, tritt ein Kurator, der algorithmische Prozesse steuert und deren Ergebnisse auswählt. Diese Verschiebung erfordert eine Neubewertung dessen, was als kreative Leistung gilt und wer Anspruch auf Urheberschaft erheben kann.

Technisch betrachtet lässt sich festhalten: KI-generierte Werke sind das Resultat eines verteilten, oft intransparenten Prozesses, der eine klare Zuschreibung von Autorschaft erschwert. Diese technische Rekonzeption der Autorschaft wirft jedoch nicht nur Fragen nach der praktischen Machbarkeit auf, sondern berührt fundamentale ethische Dimensionen. Wenn die Grenzen zwischen menschlicher und maschineller Kreativität verschwimmen, stellt sich die Frage nach moralischer Verantwortung, nach dem Eigentum an Gedanken und Ideen sowie nach den ethischen Prinzipien, die eine gerechte Anerkennung kreativer Leistungen ermöglichen.
	\section{Ethische Perspektive: Moralische Verantwortung und geistige Integrität}
	\section{Rechtliche Perspektive: Schutz geistiger Leistung und Daten im KI-Kontext}

Die rechtliche Bewertung von KI-generierten Inhalten ist für diese Arbeit vor allem dort relevant, wo sie in technische Anforderungen übersetzbar wird. Aus Sicht der Informatik sind Rechtstexte weniger \enquote{nur} Restriktionen, sondern beschreiben Randbedingungen für Systemdesign, Datenflüsse, Dokumentation und Verantwortlichkeiten. Dieses Kapitel skizziert daher die urheberrechtliche Grundlinie zur menschlichen Urheberschaft und Implikationen des AI Act für Transparenz und Governance\cite[vgl.][]{eu_ai_act_2024}.

\subsection{Urheberrechtliche Grundlinie: menschliche Urheberschaft und Nachweisbarkeit}
Das deutsche Urheberrecht schützt nach § 2 Abs. 2 UrhG \enquote{nur persönliche geistige Schöpfungen}. Der Urheber ist gemäß § 7 UrhG \enquote{der Schöpfer des Werkes}, also eine natürliche Person. Ein KI-System kann daher nicht selbst Urheber sein.

Für KI-gestützte Kreativität ist technisch vor allem die Konsequenz entscheidend: Urheberrechtlicher Schutz folgt nicht automatisch aus der Nutzung eines Tools, sondern hängt davon ab, ob ein individueller, gestaltender menschlicher Beitrag am konkreten Ergebnis nachvollziehbar ist (z.\,B. durch Auswahl-, Bearbeitungs- oder Kompositionsentscheidungen).

Ein prominentes Fallbeispiel ist das Bild \enquote{Théâtre D’opéra Spatial} von Jason Allen, bei dem das US Copyright Office Urheberrechtsschutz verneinte, weil es an hinreichender menschlicher Urheberschaft fehle\cite[vgl.][]{CopyrightGov}. Auch wenn die Rechtslage in den USA nicht unmittelbar auf Deutschland übertragbar ist, illustriert der Fall den allgemeinen Trend: Je autonomer ein System das Ergebnis prägt, desto schwieriger wird die Zuschreibung.

Für technische Kontexte folgt daraus: Wenn Organisationen KI kreativ einsetzen, werden Prozessartefakte (Prompt-Versionen, Iterationen, Auswahlentscheidungen, Nachbearbeitungsschritte, Modellversion) zu wichtigen Nachweisen, um menschliche Beiträge plausibel zu dokumentieren. Abzuwarten bleibt, wie Gerichte in Deutschland bzw. innerhalb der EU solche Nachweise künftig bewerten. Klar ist jedoch, dass Promopts ohne weitere Bearbeitung oder Auswahlentscheidungen in der Regel nicht als hinreichende menschliche Urheberschaft gelten dürften.

\subsection{EU-KI-Verordnung (AI Act): Transparenz, Dokumentation und Governance}
Der AI Act ergänzt die Urheberrechtsfrage um einen risikobasierten Regulierungsrahmen für KI-Systeme. Für generative KI und General-Purpose AI Models (GPAI) sind insbesondere Pflichten relevant, die sich in technische Umsetzungsaufgaben übersetzen lassen\cite[vgl.][]{eu_ai_act_2024}.

Aus technischer Perspektive sind vier Umsetzungsfelder zentral:
\begin{enumerate}
	\item \textbf{Kennzeichnung und Herkunft:} Mechanismen, die KI-generierte Inhalte als solche erkennbar machen (z.\,B. UI-Hinweise, Metadaten, Watermarking-Ansätze) – besonders in Kontexten mit Täuschungs- oder Desinformationsrisiken.
	\item \textbf{Dokumentation und Nachweisfähigkeit:} Versionierung von Modellen, Datenpipelines und Evaluationsartefakten sowie technische Dokumentation (z.\,B. Model Cards/Datasheets) unterstützen die Nachvollziehbarkeit von Fähigkeiten, Grenzen und Updates.
	\item \textbf{Transparenz über Trainingsdaten:} Zusammenfassungen zur Art der Trainingsdaten und Datenquellen helfen Governance, Audits und die Bewertung von Risiken (z.\,B. Bias, Datenherkunft, urheberrechtliche Konflikte).
	\item \textbf{Betriebliches Risikomanagement:} Monitoring, Incident-Handling und klare Verantwortlichkeiten (Anbietende/Betreibende/Nutzende) werden Teil des Systems, nicht nur des \enquote{Drumherums}.
\end{enumerate}

Für die Autorschaftsdebatte bedeutet das weniger eine rechtliche \enquote{Neudefinition des Autors} als eine stärkere Verpflichtung zur Transparenz über maschinelle Anteile. Technisch gewinnen damit Provenienz- und Prozessinformationen an Bedeutung (z.\,B. Modellversion, Erzeugungsmodus, Bearbeitungsschritte), um Verantwortung und Integrität nachvollziehbar zu halten.
Weiterhin unklar bleibt jedoch, welche Quellen genau im Einzelfall zur Generierung eines Outputs beigetragen haben – eine Herausforderung, die technische Transparenzmethoden (XAI) nur teilweise adressieren können. Dadurch bestetht weiterhin die Gefahr struktureller Anerkennungsdefizite für kreative Vorleistungen, die in Trainingsdaten genutzt werden.

\subsection{Zwischenfazit: Rechtliche Neudefinition geistiger Schöpfung}
Die rechtliche Betrachtung zeigt, dass das Urheberrecht an einem personenzentrierten Autorschaftsverständnis festhält, auch wenn die faktische Werkgenese zunehmend technisch vermittelt ist. Diese normative Fixierung dient der Rechtssicherheit, kann jedoch die Komplexität verteilter kreativer Prozesse nur eingeschränkt abbilden.

Aus rechtlicher Perspektive wird deutlich, dass Gesetze nicht primär beschreiben, wie kreative Leistungen entstehen, sondern Ordnung schaffen, indem sie Verantwortung, Schutz und Verwertungsrechte eindeutig zuweisen. Dabei entstehen Spannungen zu technischen und ethischen Perspektiven, die Autorschaft relationaler und gradueller denken. Auch sind diese Gesetzte tehnologieoffen gestaltendet, sodass sie zwar Rahmenbedingungen setzen, aber dem technischen Wandel nicht im Wege stehen.

Autorschaft fungiert rechtlich somit weniger als Abbild kreativer Realität und mehr als stabilisierende Kategorie. Die verbleibenden Differenzen zu technischen Beschreibungen und ethischen Bewertungen markieren keine Fehlstellen einzelner Disziplinen, sondern verweisen auf die Notwendigkeit einer interdisziplinären Gesamtsicht, wie sie im abschließenden Synthesekapitel vorgenommen wird.

	\section{Interdisziplinäre Synthese: Autorschaft im Zeitalter generativer KI}
Die vorliegende Arbeit hat Autorschaft im Kontext generativer Künstlicher Intelligenz aus philosophischer, technischer, ethischer und rechtlicher Perspektive untersucht. In der interdisziplinären Synthese werden diese Perspektiven nicht lediglich zusammengeführt, sondern in ihrem Zusammenspiel reflektiert. Ziel ist es, ein übergreifendes Verständnis von Autorschaft zu entwickeln, das der veränderten Werkgenese durch KI-Systeme gerecht wird, ohne zentrale normative Ansprüche aufzugeben.

\subsection{Autorschaft zwischen Subjekt, Prozess und Zuschreibung}
Die philosophischen Grundlagen haben gezeigt, dass Autorschaft historisch eng mit dem schöpferischen Subjekt verbunden ist. Besonders in der neuzeitlichen und aufklärerischen Tradition wird der Autor als intentional handelnde Person verstanden, deren Werk Ausdruck individueller Geistestätigkeit ist. Autorschaft ist hier nicht bloß eine funktionale Rolle, sondern an Verantwortung, Originalität und Anerkennung gebunden.

Die technische Analyse stellt dieses Verständnis grundlegend infrage. KI-generierte Werke entstehen nicht aus bewusster Intentionalität, sondern aus statistischen Prozessen, die auf Trainingsdaten, Modellarchitekturen und Nutzereingaben beruhen. Die Werkgenese ist verteilt: Entwickler, Datenlieferanten, Modell und Nutzende tragen jeweils in unterschiedlicher Weise zum Ergebnis bei. Autorschaft erscheint hier weniger als Eigenschaft eines Subjekts, sondern als Ergebnis eines komplexen soziotechnischen Prozesses.

In der Synthese wird deutlich, dass diese beiden Perspektiven nicht einfach miteinander konkurrieren, sondern unterschiedliche Ebenen adressieren. Während die Technik beschreibt, wie Werke entstehen, fragt die Philosophie danach, wem Autorschaft sinnvoll zugeschrieben werden kann. Die Spannung zwischen subjektgebundener Autorschaft und prozessualer Werkgenese ist damit kein Widerspruch, sondern Ausdruck eines Perspektivenwechsels.

\subsection{Verteilte Kreativität und normative Leerstellen}
Die technische Perspektive legt offen, dass kreative Leistungen im KI-Kontext nicht mehr eindeutig lokalisierbar sind. Diese deskriptive Einsicht erzeugt jedoch normative Leerstellen: Aus der Tatsache verteilter Beiträge folgt noch keine Antwort auf Fragen der Verantwortung, Anerkennung oder Fairness.

Hier setzt die ethische Perspektive an. Autorschaft wird nicht primär als Eigentumsanspruch, sondern als Bündel moralischer Beziehungen verstanden. Zentrale normative Forderungen betreffen Transparenz, Anerkennung von Vorleistungen und faire Beteiligung an Wertschöpfung. Besonders problematisch ist, dass Trainingsdaten häufig ohne Wissen oder Zustimmung der Urheber verwendet werden und ihre Beiträge im Output unsichtbar bleiben.

Die Synthese zeigt, dass Ethik dort interveniert, wo technische Beschreibungen enden. Die Black-Box-Problematik ist nicht nur ein technisches Transparenzproblem, sondern ein moralisches: Wenn Herkunft, Einfluss und Nähe zu bestehenden Werken nicht nachvollziehbar sind, wird Anerkennung strukturell erschwert. Autorschaft fungiert hier weniger als exklusive Zuschreibung, sondern als Orientierungsbegriff für verantwortungsvolle Praxis im kreativen Ökosystem.

\subsection{Rechtliche Fixierung und ihre Grenzen}
Die rechtliche Perspektive schließlich stabilisiert Autorschaft normativ, indem sie sie an die menschliche Person bindet. Das geltende Urheberrecht erkennt nur natürliche Personen als Urheber an und erklärt rein KI-generierte Werke weitgehend für gemeinfrei. Diese Regelung schafft Rechtssicherheit, steht jedoch in einem Spannungsverhältnis zur faktischen Werkgenese.

In der interdisziplinären Zusammenschau wird deutlich, dass das Recht bewusst vereinfacht, wo Technik und Ethik komplexe Verhältnisse beschreiben. Diese Vereinfachung ist funktional notwendig, führt aber zu Friktionen: Moralische Ansprüche auf Anerkennung oder Kompensation lassen sich nicht immer rechtlich abbilden. Die EU-KI-Verordnung reagiert auf diese Spannung mit Transparenz- und Kennzeichnungspflichten, ohne den Autorschaftsbegriff selbst grundlegend neu zu definieren.

Autorschaft erscheint rechtlich damit weniger als Beschreibung kreativer Realität, sondern als normativer Ankerpunkt zur Ordnung von Verantwortung, Haftung und Verwertung. Die Synthese macht sichtbar, dass rechtliche Zuschreibung nicht deckungsgleich mit technischer oder ethischer Bewertung ist und auch nicht sein muss.

\subsection{Autorschaft als relationale Kategorie}
Aus der Zusammenführung aller Perspektiven ergibt sich ein zentrales Ergebnis: Autorschaft im KI-Zeitalter lässt sich weder sinnvoll auf das menschliche Subjekt allein noch auf das technische System übertragen. Stattdessen erweist sie sich als relationale Kategorie, die unterschiedliche Rollen, Beiträge und Verantwortlichkeiten integriert.

Technisch ist Autorschaft verteilt, ethisch ist sie verantwortungsgebunden, rechtlich ist sie personenzentriert. Diese Differenzen sind kein Defizit, sondern Ausdruck unterschiedlicher normativer Funktionen. Autorschaft fungiert damit weniger als eindeutige Zuschreibung denn als Schnittstelle zwischen Beschreibung, Bewertung und Regulierung.

Die interdisziplinäre Synthese legt nahe, Autorschaft nicht aufzugeben, sondern neu zu kontextualisieren: nicht als exklusives Eigentum, sondern als Orientierungsbegriff für Transparenz, Anerkennung und faire Praxis in einem zunehmend hybriden kreativen Prozess.
	Die vorliegende Arbeit ging der Frage nach, wie sich das Verständnis von Autorschaft im Zeitalter generativer Künstlicher Intelligenz verändert, wenn philosophische Konzepte, technische Prozesse, moralische Verantwortung und rechtliche Zuschreibung zusammengedacht werden. Ausgangspunkt war die Beobachtung, dass KI-Systeme zunehmend Inhalte erzeugen, die traditionell als Ausdruck individueller Kreativität galten, und damit etablierte Vorstellungen von Autorschaft, Originalität und Verantwortung unter Druck geraten.

Die philosophische Analyse hat gezeigt, dass Autorschaft historisch keineswegs ein statisches Konzept ist. Während aufklärerische und ästhetische Positionen – etwa bei Kant – Autorschaft eng an das schöpferische Individuum, Intentionalität und Verantwortungsfähigkeit binden, problematisieren Autoren wie Barthes und Foucault diese Vorstellung grundlegend. Autorschaft erscheint hier weniger als Eigenschaft einer Person denn als kulturelle Zuschreibungs- und Ordnungsfunktion. Diese Perspektiven erweisen sich als besonders anschlussfähig für die Analyse KI-gestützter Kreativität, da sie Autorschaft bereits vor der Digitalisierung von einer rein individuellen Urheberschaft gelöst haben.

Die technische Betrachtung generativer KI verdeutlicht, dass KI-Outputs aus statistischen, daten- und regelbasierten Prozessen hervorgehen, denen es an Intentionalität, Selbstbezug und Verantwortungsfähigkeit fehlt. Kreative Ergebnisse entstehen in verteilten Mensch-Maschine-Konstellationen, in denen Trainingsdaten, Modellarchitektur, Entwicklerentscheidungen und Nutzereingaben zusammenwirken. Diese verteilte Werkgenese erschwert eindeutige Zuschreibungen von Autorschaft und macht deutlich, dass KI nicht sinnvoll als autonomer Autor verstanden werden kann, sondern als Werkzeug innerhalb komplexer soziotechnischer Prozesse.

Die ethische Analyse hat daran anknüpfend gezeigt, dass Autorschaft im KI-Kontext weniger als exklusiver Eigentumsanspruch denn als moralisch aufgeladene Beziehung zu verstehen ist. Zentrale normative Anforderungen betreffen Transparenz, Anerkennung kreativer Vorleistungen und faire Verfahren im Umgang mit KI-gestützter Kreativität. Besonders problematisch ist die strukturelle Unsichtbarkeit der Beiträge vieler Urheberinnen und Urheber, deren Werke als Trainingsdaten dienen, ohne dass sie informiert, anerkannt oder beteiligt werden. Autorschaft fungiert hier vor allem als Orientierungsbegriff für verantwortungsvolle Praxis, nicht als eindeutige Zuschreibung.

Die rechtliche Perspektive schließlich zeigt, dass das geltende Urheberrecht bewusst an einem personengebundenen Autorschaftsverständnis festhält. Diese normative Fixierung schafft Rechtssicherheit, kann jedoch die faktische Komplexität KI-gestützter Werkgenese nur begrenzt abbilden. Mit dem AI Act reagiert der europäische Gesetzgeber weniger durch eine Neudefinition von Autorschaft als durch Transparenz-, Dokumentations- und Governance-Pflichten, die technische Systeme so gestalten sollen, dass Verantwortung und Integrität zumindest nachvollziehbar bleiben.

In der Gesamtschau wird deutlich, dass Autorschaft im Zeitalter generativer KI weder sinnvoll auf das menschliche Subjekt allein noch auf technische Systeme übertragen werden kann. Stattdessen erweist sie sich als relationale Kategorie, die je nach Perspektive unterschiedliche Funktionen erfüllt: technisch als Beschreibung verteilter Prozesse, ethisch als Maßstab verantwortungsvoller Praxis und rechtlich als stabilisierende Zuschreibungsfigur. Die Arbeit kommt somit zu dem Ergebnis, dass Autorschaft nicht aufgegeben, sondern neu kontextualisiert werden muss – nicht als exklusives Eigentum, sondern als normativer Bezugspunkt für Transparenz, Anerkennung und faire Gestaltung hybrider kreativer Prozesse.
	% ###############################################################
	
	\newpage
	\pagenumbering{Roman}
	\setcounter{page}{\theromanBeginningEnd} % Römische Seitenzahlen fortsetzen
	\setcounter{secnumdepth}{0} % Nummerierung der Überschriften entfernt
	\section{Quellenverzeichnis}
	\setcounter{secnumdepth}{3} % Überschriften wieder numerieren (Für den Anhang)
	\printbibliography[heading=none]
	\clearpage
	\setcounter{secnumdepth}{0} % Nummerierung der Überschriften entfernt
	\section{Eigenständigkeitserklärung} 

	Mit meiner Unterschrift versichere ich, dass ich die hier vorliegende Arbeit 
	selbständig, ohne fremde Hilfe und nur mit den angegebenen Hilfsmitteln verfasst 
	habe und meine Angaben zu den verwendeten Quellen der Wahrheit entsprechen und 
	vollständig sind. Alle Quellen, aus denen ich wörtlich oder sinngemäß übernommen 
	habe, habe ich als solche gekennzeichnet.\par  

	Darüber hinaus versichere ich, dass ich sämtliche Teile der vorliegenden Arbeit, die 
	unter Zuhilfenahme künstlicher Intelligenz (KI) generiert wurden, als solche gekenn
	zeichnet habe und deren Entstehung in einer beigefügten Prozessdokumentation 
	nachgewiesen habe.\par

	Ich habe zur Kenntnis genommen, dass zuwiderlaufendes Verhalten als Täuschungs
	versuch gewertet wird und zu den in der geltenden Prüfungsverfahrensordnung 
	genannten Konsequenzen führen wird. 
	\clearpage

	% ============================================================
% KI-Prozessdokumentation
% ============================================================

\section{Prozessdokumentation zur Nutzung Künstlicher Intelligenz}
\label{sec:ki_prozessdokumentation}

\subsection{Allgemeine Angaben}

\begin{itemize}
  \item \textbf{Verwendete KI:} \emph{ChatGPT}
  \item \textbf{Anbieter:} \emph{OpenAI}
  \item \textbf{Modell / Version:} \emph{GPT-5.1}
  \item \textbf{Zugriffsart:} \emph{Web}
  \item \textbf{Abrufdatum / Nutzungszeitraum:} \emph{30.11.2025-23.12.2025}
  \item \textbf{Sprache der Interaktion:} \emph{Deutsch}
\end{itemize}

\subsection{Ziel und Zweck der KI-Nutzung}

Die KI wurde unterstützend eingesetzt, um
\begin{itemize}
  \item strukturelle Vorschläge für Kapitel und Unterkapitel zu erhalten,
  \item Formulierungen zu überarbeiten und zu präzisieren,
  \item argumentative Übergänge zu entwickeln,
  \item komplexe Inhalte verständlich zusammenzufassen.
\end{itemize}

Die fachliche Bewertung, Auswahl, Überarbeitung und finale Verantwortung der Inhalte
lagen vollständig bei den Autor:innen der Arbeit.

\subsection{Tabellarische Dokumentation der KI-Nutzung}

\renewcommand{\arraystretch}{1.3}

\begin{center}
\begin{tabular}{|p{0.23\linewidth}|p{0.27\linewidth}|p{0.27\linewidth}|p{0.18\linewidth}|}
\hline
\textbf{Anfrage an die KI (Prompt)} &
\textbf{Grund für die Verwendung der KI} &
\textbf{Bewertung der Qualität des KI-Outputs} &
\textbf{Verwendete Software / Version} \\
\hline

\emph{Kannst du zu folgender wissenschaftlicher Frage eine Gliederung einer wissenschaftlichen Arbeit generieren? Wie lässt sich geistige Autorschaft technisch, ethisch und rechtlich im KI-Zeitalter neu denken?} &
\emph{erste Strukturierung der Arbeit} &
\emph{inhaltlich geeignet, gut um erste Ideen zu sammeln} &
\emph{ChatGPT 5.1} \\
\hline

\emph{Welche Texte und Quellen kann ich gut nutzen um den Punkt x zu beschreiben?} &
\emph{Literaturfindung} &
\emph{gute, brauchbare, wissenschaftliche Quellen; allerdings nur für den ersten Überblick (hauptsächlich Lehrbücher oder bekannte philosophische Werke)} &
\emph{ChatGPT 5.1} \\
\hline

\emph{Erläutere mir die Deontologie nach Immanuel Kant und den Utilitarismus nach John Stuart Mill, Generell und im Kontext von Autorschaft. Sodass ich diesen vor dem Bezug auf das richtige Thema erklären kann} &
\emph{Erläuterung komplexer und sehr langer Texte} &
\emph{gute, zielfürende Zusammenfassung} &
\emph{ChatGPT 5.1} \\
\hline

\emph{[Prompt hier einfügen]} &
\emph{[z.\,B. Zusammenfassung theoretischer Positionen]} &
\emph{[z.\,B. korrekt, jedoch quellengeprüft]} &
\emph{[z.\,B. ChatGPT, GPT-4]} \\
\hline

\end{tabular}
\end{center}

\subsection{Qualitätssicherung und Eigenleistung}

Alle durch die KI generierten Inhalte wurden
\begin{itemize}
  \item fachlich überprüft,
  \item sprachlich überarbeitet,
  \item argumentativ eingeordnet,
  \item mit wissenschaftlicher Literatur abgeglichen.
\end{itemize}

Die Konzeption der Arbeit, die Auswahl der Literatur, die Argumentationsführung
sowie die finale Textfassung stellen eine eigenständige wissenschaftliche Leistung dar.

\subsection{Reflexion der KI-Nutzung}

Die Nutzung generativer KI erfolgte als unterstützendes Werkzeug und nicht als
Ersatz für wissenschaftliches Arbeiten. Mögliche Risiken wie inhaltliche
Ungenauigkeiten oder vereinfachende Darstellungen wurden durch kritische Prüfung
und Nachbearbeitung minimiert.
\newpage

\section{Aufteilung der Inhalte}

\begin{table}[H]
\begin{tabular}{|l|l|}
\hline
Kapitel & Verfasser \\
Einleitung & Malte Wendeler \\
Grundlagen Ethik & Malte Wendeler \\
Grundlagen KI & Max Meteling \\
TechnischePerspektive & Max Meteling \\
4.1 & Max Meteling und Malte Wendeler \\
4.2 & Max Meteling und Malte Wendeler \\
4.2.3 & Max Meteling und Malte Wendeler \\
EthischePerspektive & Max Meteling \\
RechtlichePerspektive & Max Meteling \\
InterdisziplinäreSynthese & Malte Wendeler \\
Fazit & Malte Wendeler \\
\hline
% Tragen Sie hier Ihre Aufteilung ein
\hline
\end{tabular}
\end{table}

\end{document}
