\section{Ethische Perspektive: Moralische Verantwortung und geistige Integrität}

Das vorherige Kapitel hat gezeigt, wie KI-generierte Inhalte entstehen und warum dabei oft eine verteilte Urheberschaft naheliegt. Dieses Kapitel fragt nach der ethischen Bewertung: Welche moralischen Ansprüche sind mit Autorschaft verbunden, wenn kreative Ergebnisse zunehmend über statistische Modelle zustande kommen?

Im Mittelpunkt stehen drei Fragen: (1) In welchem Sinn kann man von Eigentum an Ideen oder Stilen sprechen? (2) Welche Maßstäbe gelten für KI-gestützte Kreativität zwischen individueller Autonomie und gesellschaftlichem Nutzen? (3) Wie lassen sich Plagiat, Fairness und Anerkennung beurteilen, wenn KI-Systeme auf viele Vorleistungen zurückgreifen, die für Nutzende kaum sichtbar sind?

\subsection{Eigentum an Gedanken und Ideen}

Der Begriff \enquote{geistiges Eigentum} suggeriert, dass Ideen oder Ausdrucksformen ähnlich wie materielle Güter exklusiv kontrolliert werden können. Ethisch trägt diese Analogie nur begrenzt, weil Ideen nicht rivalisierend sind \cite{Floridi_EthicsOfInformation_2013}. Dennoch kann ihre Nutzung durch Dritte Nachteile erzeugen, etwa durch entgangene Anerkennung oder schlechtere Verwertungschancen \cite{Coeckelbergh_AI_Ethics_2020}.

Damit verschiebt sich die moralische Frage: weniger Besitz an Ideen, mehr Ansprüche aus kreativer Arbeit \cite{Tod_des_Autors}. Typisch sind Anerkennung, begrenzte Kontrolle (z.B. Schutz vor Täuschung) und in manchen Fällen Kompensation, wenn Dritte systematisch wirtschaftlich profitieren.

\subsubsection{Ideen, Stile und Ausdruck: Was ist moralisch schutzwürdig?}

Für die Bewertung hilft eine Unterscheidung zwischen abstrakten Ideen, konkreten Ausdrucksformen und Stilen. Abstrakte Ideen gehören meist zum gemeinsamen kulturellen Vorrat; eine exklusive Zuordnung würde kulturelle Weiterentwicklung eher bremsen \cite{Tod_des_Autors}.

Konkrete Ausdrucksformen sind enger an die Arbeit (und oft an die Person) gebunden \cite{Kant_MetaphysikDerSitten_1797}. Stile liegen dazwischen: Sie sind erlernt und geteilt, zugleich aber für viele Kreative zentral. Generative KI verschärft die Spannung, weil Stilmerkmale schnell und massenhaft reproduziert werden können. Problematisch wird das besonders, wenn Stilkopien Verwechslung nahelegen oder von der Reputation einer Person profitieren\cite{Mittelstadt_EthicsOfAlgorithms_2016}.

\subsubsection{Kollektive Wissensproduktion als Vergleichsmaßstab}

Ein zweiter Blick ergibt sich aus kollektiver Wissens- und Kulturproduktion (z.B. Open Source oder Remix-Kulturen). Dort ist Nutzung häufig erlaubt, aber an Regeln gebunden, etwa Namensnennung oder Transparenz\cite{Tod_des_Autors}.

KI-Training baut ebenfalls auf Beiträgen vieler auf, allerdings meist ohne bewusstes Kooperationsverhältnis. Inhalte werden häufig ohne Wissen der Urheber in Trainingsdaten aufgenommen. Daraus ergibt sich ein Legitimationsproblem: Selbst wenn Outputs keine 1:1-Kopien sind, profitieren Systeme von fremden Vorleistungen, ohne dass Betroffene verlässlich informiert, beteiligt oder bei kommerzieller Nutzung angemessen kompensiert werden\cite{Coeckelbergh_AI_Ethics_2020,Mittelstadt_EthicsOfAlgorithms_2016}.

\subsection{Ethik der Kreativität}

Um moralische Maßstäbe zu ordnen, helfen drei Perspektiven: deontologische Ethik (Pflichten und Rechte), utilitaristische Abwägungen (Folgen und Nutzen) und eine informationsethische Perspektive (KI als Teil eines Informationsökosystems).

\subsubsection{Deontologische Sicht: Respekt vor geistiger Autonomie}

Deontologische Ansätze betonen, dass Menschen nicht bloß als Mittel für fremde Zwecke behandelt werden sollen \cite{Kant_MetaphysikDerSitten_1797}. Übertragen auf kreative Arbeit heißt das: Schaffende sind nicht nur \enquote{Datenlieferanten}. Wenn ihre Inhalte ohne Wissen und Zustimmung in großem Stil für kommerzielle Systeme genutzt werden, wirkt das schnell wie Instrumentalisierung.

Zugleich ist kulturelle Bezugnahme nicht per se unethisch; Kunst und Wissenschaft leben von Zitaten und Weiterentwicklungen. Problematisch wird es dort, wo Nutzung verdeckt geschieht und Herkunft unsichtbar bleibt\cite{Coeckelbergh_AI_Ethics_2020,Winter_WarumKIKeineSchoneKunst_2022,Foucault_WasIstEinAutor_1969}. Ein pragmatisches Minimalkriterium ist daher: Transparenz (soweit möglich), keine Täuschung und eine Praxis, welche Wert auf Anerkennung legt.

\subsubsection{Utilitaristische Sicht: Gesellschaftlicher Nutzen vs. individueller Verlust}

Utilitaristische Ansätze fragen nach Folgen. Generative KI kann Hürden senken und neue Ausdrucksformen ermöglichen (z.\,B. mehrsprachige Texte, Visualisierungen ohne Spezialausbildung) und damit gesellschaftlichen Nutzen stiften.

Dem stehen Risiken gegenüber: Verdrängung, weniger Vertrauen in Originalität und Konzentration auf Wertschöpfung. Entscheidend ist die Verteilung von Nutzen und Lasten \cite{Coeckelbergh_AI_Ethics_2020,Mittelstadt_EthicsOfAlgorithms_2016}. Eine utilitaristische Rechtfertigung ist daher stärker, wenn es Ausgleichsmechanismen gibt, z.B. Vergütung, realistische Opt-out/Opt-in-Optionen oder verlässliche Kennzeichnung und Transparenz.

\subsubsection{Daten- und Informationsethik: Verantwortung im Informationsökosystem}

Eine dritte Perspektive betrachtet generative KI als Teil eines Informationsökosystems, in dem Datenflüsse, Macht und Verantwortung neu verteilt werden. Kreative Inhalte werden dabei auch zum Rohstoff des Lernens.

Zentral sind zwei Punkte: epistemische Integrität (Nachvollziehbarkeit von Entstehung und Quellenlage) und Governance (wer entscheidet über Trainingsdaten, Grenzen, Schutzmechanismen). Je stärker diese Entscheidungen bei privaten Plattformen liegen, desto wichtiger sind transparente Regeln und nachvollziehbare Standards\cite{Floridi_EthicsOfInformation_2013,Mittelstadt_EthicsOfAlgorithms_2016}.

\subsection{Plagiat, Fairness und Anerkennung}

Viele Debatten über KI-Kreativität laufen auf den Plagiatsvorwurf hinaus. Wichtig ist: Plagiat ist nicht nur eine juristische Kategorie, sondern auch ein moralischer Vorwurf\cite{Kant_MetaphysikDerSitten_1797,Floridi_EthicsOfInformation_2013}.

\subsubsection{Ab wann wird algorithmische Inspiration zu Ideendiebstahl?}

Inspiration gehört zu kreativer Arbeit dazu~\cite{Boden_MindsAndMachines_2009}. Moralisch problematisch wird Aneignung vor allem dann, wenn Übernahmen sehr spezifisch sind oder ihre Herkunft verschleiert wird\cite{Floridi_EthicsOfInformation_2013}.

Bei KI-Systemen ist die Bewertung schwieriger, weil Nutzende oft nicht einschätzen können, wie nah ein Output an bestehenden Werken liegt. Die Verantwortung verteilt sich daher auf mehrere Akteure: Modellbetreiber (Schutzmechanismen/Transparenz), Nutzende (Prüfung/Kennzeichnung) und Organisationen (klare Regeln im professionellen Einsatz)\cite{Coeckelbergh_AI_Ethics_2020,Mittelstadt_EthicsOfAlgorithms_2016}.

\subsubsection{Stilkopien, Remixing und algorithmische Collage}

Stilkopien sind ein Grenzfall: Sie können Hommage sein, aber auch Reputation ausnutzen. Entscheidend sind Intention und Kontext \cite{Winter_WarumKIKeineSchoneKunst_2022}. Täuschungsnah wird es besonders dann, wenn Stil gezielt als Ersatzprodukt eingesetzt wird (\enquote{im Stil von X}) und Verwechslung nahelegt.

Remixing und Collage sind moralisch eher akzeptiert, wenn Transformation erkennbar ist und Bezugnahmen nicht verschleiert werden\cite{Foucault_WasIstEinAutor_1969}. KI-Outputs wirken oft transformativ, lösen sich aber von sichtbarer Quellenpraxis; dadurch entsteht ein Anerkennungsproblem.

\subsubsection{Fairness als Anerkennungs- und Verfahrensfrage}

Fairness betrifft in diesem Kontext nicht nur das Ergebnis (wer bekommt Geld oder Ruhm), sondern auch das Verfahren (wie kommen Systeme zu ihren Outputs). Drei Fairness-Dimensionen sind besonders relevant:

\begin{enumerate}
	\item \textbf{Anerkennungsfairness:} Kreative Vorleistungen sollten nicht strukturell unsichtbar gemacht werden. Wo individuelle Zuschreibung nicht möglich ist, sind zumindest transparente Regeln und klare Kennzeichnung wichtig.
	\item \textbf{Kompensationsfairness:} Wenn aus kreativen Ökosystemen systematisch Wert extrahiert wird, stellt sich die Frage nach Beteiligungs- oder Vergütungsmodellen, insbesondere bei kommerzieller Nutzung.
	\item \textbf{Teilhabefairness:} Kreative sollten reale Optionen haben, sich zu beteiligen oder sich zu entziehen (Opt-in/Opt-out), und ihre Interessen sollten in Governance-Strukturen berücksichtigt werden.
\end{enumerate}
\cite{Floridi_EthicsOfInformation_2013,Coeckelbergh_AI_Ethics_2020,Mittelstadt_EthicsOfAlgorithms_2016}
Für akademische und professionelle Kontexte folgt daraus ein Integritätsgebot: KI-Nutzung sollte so dokumentiert werden, dass Eigenleistung und maschinelle Unterstützung nachvollziehbar bleiben. Ziel ist weniger ein pauschales Verbot als die Vermeidung von Täuschung und eine möglichst faire Anerkennung.

\subsection{Zwischenfazit: Autorschaft als moralische Verantwortung}

Die ethische Analyse hat gezeigt, dass Autorschaft im KI-Kontext weniger als exklusiver Besitzanspruch zu verstehen ist, sondern als moralisch aufgeladene Beziehung zwischen Schaffenden, Nutzenden und technischen Systemen. Zentrale normative Forderungen betreffen Transparenz, Anerkennung kreativer Vorleistungen und faire Verfahren im Umgang mit KI-gestützter Kreativität.

KI verschiebt die Bedingungen, weil Vorleistungen zu Trainingsressourcen werden und Quellenbezüge im Output oft unsichtbar bleiben. Die zentrale Aufgabe ist daher weniger, KI zum \enquote{Autor} zu erklären, sondern Verantwortlichkeiten im Ökosystem zu klären (Betreiber, Nutzende, Schaffende) und Regeln für Transparenz, Kennzeichnung und Integrität zu entwickeln.

Diese Aufgabe steht jedoch in einem Spannungsverhältnis zur technischen Realität. Die Intransparenz vieler Modelle erschwert es, individuelle Beiträge sichtbar zu machen, und begünstigt damit strukturelle Anerkennungsdefizite. Autorschaft wird so zu einem normativ anspruchsvollen, aber praktisch schwer einlösbaren Konzept.

Zugleich zeigt sich, dass moralische Ansprüche nicht automatisch rechtlich abgesichert sind. Während ethische Überlegungen differenzierte Verantwortlichkeiten nahelegen, operiert das geltende Urheberrecht mit klaren, personengebundenen Zuschreibungen. Die folgende rechtliche Analyse greift diese Spannung auf und untersucht, wie Autorschaft im bestehenden Rechtsrahmen fixiert wird und wo dabei Regelungslücken entstehen.