Die vorliegende Arbeit ging der Frage nach, wie sich das Verständnis von Autorschaft im Zeitalter generativer Künstlicher Intelligenz verändert, wenn philosophische Konzepte, technische Prozesse, moralische Verantwortung und rechtliche Zuschreibung zusammengedacht werden. Ausgangspunkt war die Beobachtung, dass KI-Systeme zunehmend Inhalte erzeugen, die traditionell als Ausdruck individueller Kreativität galten, und damit etablierte Vorstellungen von Autorschaft, Originalität und Verantwortung unter Druck geraten.

Die philosophische Analyse hat gezeigt, dass Autorschaft historisch keineswegs ein statisches Konzept ist. Während aufklärerische und ästhetische Positionen – etwa bei Kant – Autorschaft eng an das schöpferische Individuum, Intentionalität und Verantwortungsfähigkeit binden, problematisieren Autoren wie Barthes und Foucault diese Vorstellung grundlegend. Autorschaft erscheint hier weniger als Eigenschaft einer Person denn als kulturelle Zuschreibungs- und Ordnungsfunktion. Diese Perspektiven erweisen sich als besonders anschlussfähig für die Analyse KI-gestützter Kreativität, da sie Autorschaft bereits vor der Digitalisierung von einer rein individuellen Urheberschaft gelöst haben.

Die technische Betrachtung generativer KI verdeutlicht, dass KI-Outputs aus statistischen, daten- und regelbasierten Prozessen hervorgehen, denen es an Intentionalität, Selbstbezug und Verantwortungsfähigkeit fehlt. Kreative Ergebnisse entstehen in verteilten Mensch-Maschine-Konstellationen, in denen Trainingsdaten, Modellarchitektur, Entwicklerentscheidungen und Nutzereingaben zusammenwirken. Diese verteilte Werkgenese erschwert eindeutige Zuschreibungen von Autorschaft und macht deutlich, dass KI nicht sinnvoll als autonomer Autor verstanden werden kann, sondern als Werkzeug innerhalb komplexer soziotechnischer Prozesse.

Die ethische Analyse hat daran anknüpfend gezeigt, dass Autorschaft im KI-Kontext weniger als exklusiver Eigentumsanspruch denn als moralisch aufgeladene Beziehung zu verstehen ist. Zentrale normative Anforderungen betreffen Transparenz, Anerkennung kreativer Vorleistungen und faire Verfahren im Umgang mit KI-gestützter Kreativität. Besonders problematisch ist die strukturelle Unsichtbarkeit der Beiträge vieler Urheberinnen und Urheber, deren Werke als Trainingsdaten dienen, ohne dass sie informiert, anerkannt oder beteiligt werden. Autorschaft fungiert hier vor allem als Orientierungsbegriff für verantwortungsvolle Praxis, nicht als eindeutige Zuschreibung.

Die rechtliche Perspektive schließlich zeigt, dass das geltende Urheberrecht bewusst an einem personengebundenen Autorschaftsverständnis festhält. Diese normative Fixierung schafft Rechtssicherheit, kann jedoch die faktische Komplexität KI-gestützter Werkgenese nur begrenzt abbilden. Mit dem AI Act reagiert der europäische Gesetzgeber weniger durch eine Neudefinition von Autorschaft als durch Transparenz-, Dokumentations- und Governance-Pflichten, die technische Systeme so gestalten sollen, dass Verantwortung und Integrität zumindest nachvollziehbar bleiben.

In der Gesamtschau wird deutlich, dass Autorschaft im Zeitalter generativer KI weder sinnvoll auf das menschliche Subjekt allein noch auf technische Systeme übertragen werden kann. Stattdessen erweist sie sich als relationale Kategorie, die je nach Perspektive unterschiedliche Funktionen erfüllt: technisch als Beschreibung verteilter Prozesse, ethisch als Maßstab verantwortungsvoller Praxis und rechtlich als stabilisierende Zuschreibungsfigur. Die Arbeit kommt somit zu dem Ergebnis, dass Autorschaft nicht aufgegeben, sondern neu kontextualisiert werden muss – nicht als exklusives Eigentum, sondern als normativer Bezugspunkt für Transparenz, Anerkennung und faire Gestaltung hybrider kreativer Prozesse.