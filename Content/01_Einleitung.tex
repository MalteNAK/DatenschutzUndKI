\section{Einleitung}
\subsection{Hintergrund}
Die rasante Entwicklung generativer KI-Systeme wie großer Sprach- und Bildmodelle stellt zentrale Begriffe moderner Kultur- und Geisteswissenschaften infrage. Insbesondere das Konzept der Autorschaft gerät unter Druck, wenn Texte, Bilder oder Musik scheinbar autonom von technischen Systemen erzeugt werden. Werke, die bislang als Ausdruck individueller Kreativität galten, entstehen zunehmend in komplexen Mensch-Maschine-Konstellationen, deren Beiträge sich nicht eindeutig trennen lassen. Diese Entwicklung wirft grundlegende philosophische, technische und rechtliche Fragen auf: Wer ist Autor eines KI-generierten Werkes? Welche Rolle spielen Intentionalität, Originalität und Verantwortung in einem Prozess, der auf statistischer Mustererkennung beruht? Und wie lassen sich bestehende Konzepte geistigen Eigentums und kreativer Leistung unter diesen Bedingungen sinnvoll anwenden oder neu denken? Der Diskurs um KI-gestützte Kreativität ist dabei nicht nur theoretischer Natur, sondern auch von hoher gesellschaftlicher Relevanz, etwa im Hinblick auf Urheberrecht, kulturelle Produktion, akademische Praxis und kreative Berufe.
\subsection{Zielstellung}
Die vorliegende Arbeit nähert sich diesen Fragen aus verschiedenen Perspektiven. Ausgangspunkt ist ein philosophisches Verständnis von Autorschaft, das historisch gewachsen ist und insbesondere in der Ästhetik der Aufklärung und der Moderne ausgearbeitet wurde. Dabei werden sowohl individualistische Konzepte, etwa bei Immanuel Kant, als auch kritischere Perspektiven, welche Autorschaft als kollektive Kreativität ansehen, bei Autoren wie Roland Barthes und Michel Foucault berücksichtigt. Darauf aufbauend wird untersucht, wie sich Autorschaft im Kontext algorithmischer Kreativität technisch rekonfigurieren lässt und warum KI-Systeme die Zuschreibung von Urheberschaft strukturell erschweren. Folglich werden die ethischen Implikationen dieser Veränderungen diskutiert, insbesondere im Hinblick auf Verantwortung, Anerkennung und den normativen Status geistiger Leistungen. Hierzu wird sich auf aktuelle Richtlinien und Gesetzestexte, wie beispielsweise dem AI-Act, bezogen. Schließlich werden diese Vorschriften, aus einer moralischen und technischen Sicht, basierend auf den erarbeiteten Aspekten, bewertet. 