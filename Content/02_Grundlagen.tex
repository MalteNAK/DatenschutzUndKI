\section{Grundlagen: Autorschaft und Künstliche Intelligenz}
\subsection{Historischer und philosophischer Begriff der Autorschaft}
Heutzutage, sowie seit der Neuzeit und insbesondere im Zuge der Aufklärung und der, im späten 18. Jahrhundert aufkommenden romantischen Genieästhetik, beherrscht die Rolle des Autors den Diskurs von fast jedem erdachten Werk (vgl. Barthes, \textit{Der Tod des Autors}). Im Folgenden soll das Autorschaftsverständnis untersucht werden, indem auf die Produktion eines kreativen Werks eingegangen wird. Zum Einen durch ein schöpferes Individuum oder durch die Kreation in kollektiver Form. Zudem werden die damit verbundenen Konzepte von Originalität, Inspiration und geistiges Eigentum behandelt, um den Einfluss von KI auf das Thema Autorschaft zu diskutieren.
\subsubsection{Das schöpferische Individuum}
Hier werd ich mich größtenteils auf Kant beziehen.\\
Kernaussage:\\
schöpferes Wesen (in unserem Fall der Autor), das in der Lage ist, moralische Entscheidungen zu treffen und nach moralischen Prinzipien zu handeln.\\
Autorschaft 
\subsubsection{Kollektive Kreativität}
Hier werde ich mich auf Michel Foucaults Werke "Was ist ein Autor" und "Autorfunktion", sowie Barthes "Der Tod des Autors." beziehen.\\
Kernaussage:\\
Relevanz des Autors\\
Der Autor ist nicht die letzte Quelle der Bedeutung eines Textes, diese entsteht eher im Lesen und im kulturellen Kontext.\\
Texte als Gewebe aus Zitaten, nicht Ausdruck des inneren Selbst.\\
Autor als Funktion des Diskurses und nicht als natürliche Person.\\
\textbf{Autorschaft erfüllt gesellschaftliche Aufgaben:\\
Zuschreibung\\
Verantwortungszuteilung\\
Kanonisierung\\
Kontrolle von Diskursen\\
--> Autorschaft entsteht dort, wo Zurechenbarkeit notwendig ist.\\} 
KI hat keinen Selbstbezug, keine Fähigkeit Gründe zu haben, keine Teilnahme an Diskursen als Subjekt oder Person.\\
\subsubsection{Originalität, Inspiration und geistiges Eigentum}
Weiß ich noch nicht. :/