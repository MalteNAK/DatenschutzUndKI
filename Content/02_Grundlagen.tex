\section{Grundlagen: Autorschaft und Künstliche Intelligenz}
\subsection{Historischer und philosophischer Begriff der Autorschaft}
Heutzutage, sowie seit der Neuzeit und insbesondere im Zuge der Aufklärung und der, im späten 18. Jahrhundert aufkommenden romantischen Genieästhetik, beherrscht die Rolle des Autors den Diskurs von fast jedem erdachten Werk (vgl. Barthes, \textit{Der Tod des Autors}). Im Folgenden soll das Autorschaftsverständnis untersucht werden, indem auf die Produktion eines kreativen Werks eingegangen wird. Zum Einen durch ein schöpferes Individuum oder durch die Kreation in kollektiver Form. Zudem werden die damit verbundenen Konzepte von Originalität, Inspiration und geistiges Eigentum behandelt, um den Einfluss von KI auf das Thema Autorschaft zu diskutieren.
\subsubsection{Das schöpferische Individuum}
Aus Kants Verständnis von Genie, Freiheit und Zurechnung folgt, dass Autorschaft an Bedingungen gebunden ist, die KI-Systeme nicht erfüllen.
\\
Künstliche Intelligenz stört die Diskussionen rund um Autorschaft, indem die Fragen, wer ein Werk geschaffen hat und wie es geschaffen wurde, neu betrachtet werden müssen. Nach Immanuel Kants \textit{Kritik der Urteilskraft} entwickelt sich ein Verständnis von ästhetischer Autorschaft, welche untrennbar mit dem schöpferischen Individuum verbunden ist. Ästhetische Autorschaft setzt ein intentionales und verantwortbares Handeln voraus. Somit beginnt Autorschaft nicht nur wenn ein Werk entsteht, sondern diesem Werk ein Ausdruck eines Subjekts verliehen werden kann. Ein zentrales Konzept, welches Kant in diesem Prozess beschreibt, ist das Genie. Es sei ein natürliches Talent "durch welches die Natur der Kunst die Regel gibt". Demnach wird dem Schöpfer eines Werks von Kant mehr vorrausgesetzt als bloße Regelanwendung oder Nachahmung, sondern eine originäre schöpferische Leistung. Ein solche Leistung lässt sich niht vollständig rationalisieren oder in explizite Anweisungen überführen. Des Weiteren muss jenes Genie in der Lage sein, mithilfe von Einbildungskraft und Verstand, neue ästhetische Gedanken hervorzubringen. 
\\
Hier werd ich mich größtenteils auf Kant beziehen.\\
Kernaussage:\\
schöpferes Wesen (in unserem Fall der Autor), das in der Lage ist, moralische Entscheidungen zu treffen und nach moralischen Prinzipien zu handeln.\\
\subsubsection{Kollektive Kreativität}
Hier werde ich mich auf Michel Foucaults Werke "Was ist ein Autor" und "Autorfunktion", sowie Barthes "Der Tod des Autors." beziehen.\\
Kernaussage:\\
Relevanz des Autors\\
Der Autor ist nicht die letzte Quelle der Bedeutung eines Textes, diese entsteht eher im Lesen und im kulturellen Kontext.\\
Texte als Gewebe aus Zitaten, nicht Ausdruck des inneren Selbst.\\
Autor als Funktion des Diskurses und nicht als natürliche Person.\\
\textbf{Autorschaft erfüllt gesellschaftliche Aufgaben:\\
Zuschreibung\\
Verantwortungszuteilung\\
Kanonisierung\\
Kontrolle von Diskursen\\
--> Autorschaft entsteht dort, wo Zurechenbarkeit notwendig ist.\\} 
KI hat keinen Selbstbezug, keine Fähigkeit Gründe zu haben, keine Teilnahme an Diskursen als Subjekt oder Person.\\
\subsubsection{Originalität, Inspiration und geistiges Eigentum}
Weiß ich noch nicht. :/