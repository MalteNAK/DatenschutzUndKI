\section{Grundlagen: Autorschaft und Künstliche Intelligenz}
\subsection{Historischer und philosophischer Begriff der Autorschaft}
Heutzutage, sowie seit der Neuzeit und insbesondere im Zuge der Aufklärung und der, im späten 18. Jahrhundert aufkommenden romantischen Genieästhetik, beherrscht die Rolle des Autors den Diskurs von fast jedem erdachten Werk (vgl. Barthes, \textit{Der Tod des Autors}). Im Folgenden soll das Autorschaftsverständnis untersucht werden, indem auf die Produktion eines kreativen Werks eingegangen wird. Zum Einen durch ein schöpferes Individuum oder durch die Kreation in kollektiver Form. Zudem werden die damit verbundenen Konzepte von Originalität, Inspiration und geistiges Eigentum behandelt, um den Einfluss von KI auf das Thema Autorschaft zu diskutieren.
\subsubsection{Das schöpferische Individuum}
Künstliche Intelligenz stört die Diskussionen rund um Autorschaft, indem die Fragen, wer ein Werk geschaffen hat und wie es geschaffen wurde, neu betrachtet werden müssen. Nach Immanuel Kants \textit{Kritik der Urteilskraft} entwickelt sich ein Verständnis von ästhetischer Autorschaft, welche untrennbar mit dem schöpferischen Individuum verbunden ist. Ästhetische Autorschaft setzt ein intentionales und verantwortbares Handeln voraus. Somit beginnt Autorschaft nicht nur wenn ein Werk entsteht, sondern diesem Werk ein Ausdruck eines Subjekts verliehen werden kann. Ein zentrales Konzept, welches Kant in diesem Prozess beschreibt, ist das Genie. Es sei ein natürliches Talent "durch welches die Natur der Kunst die Regel gibt". Demnach wird dem Schöpfer eines Werks von Kant mehr vorrausgesetzt als bloße Regelanwendung oder Nachahmung, sondern eine originäre schöpferische Leistung. Ein solche Leistung lässt sich niht vollständig rationalisieren oder in explizite Anweisungen überführen. Des Weiteren muss jenes Genie in der Lage sein, mithilfe von Einbildungskraft und Verstand, neue ästhetische Gedanken hervorzubringen. 

\subsubsection{Kollektive Kreativität}
Während Kant Autorschaft stark an das Individuum bindet, problematisieren Michel Foucault und Roland Barthes diese Vorstellung grundlegend. Beide lösen die Idee des Autors als originäre Quelle von Bedeutung auf, wenn auch mit unterschiedlichen Akzenten.\par

Barthes argumentiert, dass ein Text nicht als Ausdruck innerer Intentionalität des Autors zu verstehen sei, sondern als „Gewebe aus Zitaten“, das aus kulturellen, sprachlichen und historischen Bezügen besteht. Bedeutung entsteht demnach nicht im Akt des Schreibens, sondern im Lesen und im jeweiligen Kontext der Rezeption. Der Autor verliert seine privilegierte Stellung als Sinnstifter.\par

Foucault geht einen Schritt weiter, indem er den Autor nicht als natürliche Person, sondern als Funktion des Diskurses beschreibt. Die Autorfunktion erfüllt gesellschaftliche Aufgaben wie Zuschreibung, Verantwortungszuteilung, Kanonisierung und Kontrolle von Diskursen. Autorschaft entsteht dort, wo Zurechenbarkeit notwendig ist, nicht dort, wo kreative Produktion faktisch stattfindet.\par

Diese Perspektiven machen deutlich, dass kreative Werke stets in kollektiven Zusammenhängen entstehen: Sie greifen auf bestehende Diskurse zurück, reproduzieren kulturelle Muster und werden erst im sozialen Raum mit Bedeutung versehen. Kreativität ist somit nie rein individuell, sondern strukturell eingebettet.\par

Für KI bedeutet dies jedoch keine Aufwertung zur Autorschaft. Zwar operieren KI-Systeme paradigmatisch kollektiv, gespeist aus großen Datenmengen kultureller Artefakte, doch fehlt ihnen der Selbstbezug, der für die Autorfunktion zentral ist. KI kann weder Verantwortung tragen noch als diskursives Subjekt auftreten. Autorschaft bleibt daher eine menschliche Zuschreibungspraxis, auch wenn die Produktion selbst zunehmend kollektiv und technisch vermittelt ist.\par

\subsubsection{Originalität, Inspiration und geistiges Eigentum}
Die Konzepte Originalität und Inspiration sind historisch eng mit dem Ideal des schöpferischen Individuums verbunden. Bei Kant ist Originalität ein zentrales Merkmal des Genies: Ein Werk gilt als originell, wenn es nicht bloß bestehende Regeln reproduziert, sondern neue hervorbringt. Inspiration ist dabei kein mystischer Akt, sondern Ausdruck freier schöpferischer Tätigkeit.\par

Im Kontext Künstlicher Intelligenz geraten diese Begriffe unter Druck. KI-Systeme erzeugen Inhalte, die neu erscheinen, ohne originär im kantischen Sinne zu sein. Ihre Leistungen basieren auf der Rekombination vorhandener Daten, nicht auf autonomer Regelsetzung. Originalität verschiebt sich dadurch von einem ontologischen zu einem funktionalen Kriterium: Neu ist, was als neu wahrgenommen wird.\par

Hier bietet der Utilitarismus von John Stuart Mill einen ergänzenden Zugang. Mill bewertet Handlungen nicht nach ihrer inneren Motivation, sondern nach ihren Folgen. Übertragen auf kreative Werke bedeutet dies, dass der moralische und rechtliche Status eines Outputs weniger von seiner Entstehung als von seinem gesellschaftlichen Nutzen abhängt. Ein Werk ist relevant, wenn es zur Förderung von Wissen, Wohlstand oder kultureller Entwicklung beiträgt.\par

Diese Perspektive ist besonders für Fragen des geistigen Eigentums relevant. Während Kant Autorschaft als Ausdruck personaler Freiheit begreift, erlaubt Mill eine stärker pragmatische Betrachtung: Schutzrechte können gerechtfertigt sein, wenn sie insgesamt positive Folgen haben, etwa durch Anreize für kreative Produktion. Im Fall von KI stellt sich daher weniger die Frage, ob sie originell „schafft“, sondern ob und wie ihre Nutzung reguliert werden sollte, um gesellschaftlichen Nutzen zu maximieren, ohne menschliche Urheberrechte zu untergraben.\par

Originalität, Inspiration und geistiges Eigentum erscheinen somit nicht als feste Eigenschaften eines Werkes, sondern als normative Zuschreibungen, die im Spannungsfeld zwischen individueller Autorschaft, kollektiver Kreativität und gesellschaftlichem Nutzen neu ausgehandelt werden müssen.\par