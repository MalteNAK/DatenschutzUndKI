\section{Interdisziplinäre Synthese: Autorschaft im Zeitalter generativer KI}
Die vorliegende Arbeit hat Autorschaft im Kontext generativer Künstlicher Intelligenz aus philosophischer, technischer, ethischer und rechtlicher Perspektive untersucht. In der interdisziplinären Synthese werden diese Perspektiven nicht lediglich zusammengeführt, sondern in ihrem Zusammenspiel reflektiert. Ziel ist es, ein übergreifendes Verständnis von Autorschaft zu entwickeln, das der veränderten Werkgenese durch KI-Systeme gerecht wird, ohne zentrale normative Ansprüche aufzugeben.

\subsection{Autorschaft zwischen Subjekt, Prozess und Zuschreibung}
Die philosophischen Grundlagen haben gezeigt, dass Autorschaft historisch eng mit dem schöpferischen Subjekt verbunden ist. Besonders in der neuzeitlichen und aufklärerischen Tradition wird der Autor als intentional handelnde Person verstanden, deren Werk Ausdruck individueller Geistestätigkeit ist. Autorschaft ist hier nicht bloß eine funktionale Rolle, sondern an Verantwortung, Originalität und Anerkennung gebunden.

Die technische Analyse stellt dieses Verständnis grundlegend infrage. KI-generierte Werke entstehen nicht aus bewusster Intentionalität, sondern aus statistischen Prozessen, die auf Trainingsdaten, Modellarchitekturen und Nutzereingaben beruhen. Die Werkgenese ist verteilt: Entwickler, Datenlieferanten, Modell und Nutzende tragen jeweils in unterschiedlicher Weise zum Ergebnis bei. Autorschaft erscheint hier weniger als Eigenschaft eines Subjekts, sondern als Ergebnis eines komplexen soziotechnischen Prozesses.

In der Synthese wird deutlich, dass diese beiden Perspektiven nicht einfach miteinander konkurrieren, sondern unterschiedliche Ebenen adressieren. Während die Technik beschreibt, wie Werke entstehen, fragt die Philosophie danach, wem Autorschaft sinnvoll zugeschrieben werden kann. Die Spannung zwischen subjektgebundener Autorschaft und prozessualer Werkgenese ist damit kein Widerspruch, sondern Ausdruck eines Perspektivenwechsels.

\subsection{Verteilte Kreativität und normative Leerstellen}
Die technische Perspektive legt offen, dass kreative Leistungen im KI-Kontext nicht mehr eindeutig lokalisierbar sind. Diese deskriptive Einsicht erzeugt jedoch normative Leerstellen: Aus der Tatsache verteilter Beiträge folgt noch keine Antwort auf Fragen der Verantwortung, Anerkennung oder Fairness.

Hier setzt die ethische Perspektive an. Autorschaft wird nicht primär als Eigentumsanspruch, sondern als Bündel moralischer Beziehungen verstanden. Zentrale normative Forderungen betreffen Transparenz, Anerkennung von Vorleistungen und faire Beteiligung an Wertschöpfung. Besonders problematisch ist, dass Trainingsdaten häufig ohne Wissen oder Zustimmung der Urheber verwendet werden und ihre Beiträge im Output unsichtbar bleiben.

Die Synthese zeigt, dass Ethik dort interveniert, wo technische Beschreibungen enden. Die Black-Box-Problematik ist nicht nur ein technisches Transparenzproblem, sondern ein moralisches: Wenn Herkunft, Einfluss und Nähe zu bestehenden Werken nicht nachvollziehbar sind, wird Anerkennung strukturell erschwert. Autorschaft fungiert hier weniger als exklusive Zuschreibung, sondern als Orientierungsbegriff für verantwortungsvolle Praxis im kreativen Ökosystem.

\subsection{Rechtliche Fixierung und ihre Grenzen}
Die rechtliche Perspektive schließlich stabilisiert Autorschaft normativ, indem sie sie an die menschliche Person bindet. Das geltende Urheberrecht erkennt nur natürliche Personen als Urheber an und erklärt rein KI-generierte Werke weitgehend für gemeinfrei. Diese Regelung schafft Rechtssicherheit, steht jedoch in einem Spannungsverhältnis zur faktischen Werkgenese.

In der interdisziplinären Zusammenschau wird deutlich, dass das Recht bewusst vereinfacht, wo Technik und Ethik komplexe Verhältnisse beschreiben. Diese Vereinfachung ist funktional notwendig, führt aber zu Friktionen: Moralische Ansprüche auf Anerkennung oder Kompensation lassen sich nicht immer rechtlich abbilden. Die EU-KI-Verordnung reagiert auf diese Spannung mit Transparenz- und Kennzeichnungspflichten, ohne den Autorschaftsbegriff selbst grundlegend neu zu definieren.

Autorschaft erscheint rechtlich damit weniger als Beschreibung kreativer Realität, sondern als normativer Ankerpunkt zur Ordnung von Verantwortung, Haftung und Verwertung. Die Synthese macht sichtbar, dass rechtliche Zuschreibung nicht deckungsgleich mit technischer oder ethischer Bewertung ist und auch nicht sein muss.

\subsection{Autorschaft als relationale Kategorie}
Aus der Zusammenführung aller Perspektiven ergibt sich ein zentrales Ergebnis: Autorschaft im KI-Zeitalter lässt sich weder sinnvoll auf das menschliche Subjekt allein noch auf das technische System übertragen. Stattdessen erweist sie sich als relationale Kategorie, die unterschiedliche Rollen, Beiträge und Verantwortlichkeiten integriert.

Technisch ist Autorschaft verteilt, ethisch ist sie verantwortungsgebunden, rechtlich ist sie personenzentriert. Diese Differenzen sind kein Defizit, sondern Ausdruck unterschiedlicher normativer Funktionen. Autorschaft fungiert damit weniger als eindeutige Zuschreibung denn als Schnittstelle zwischen Beschreibung, Bewertung und Regulierung.

Die interdisziplinäre Synthese legt nahe, Autorschaft nicht aufzugeben, sondern neu zu kontextualisieren: nicht als exklusives Eigentum, sondern als Orientierungsbegriff für Transparenz, Anerkennung und faire Praxis in einem zunehmend hybriden kreativen Prozess.