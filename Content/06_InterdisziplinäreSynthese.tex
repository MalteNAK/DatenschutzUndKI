\section{Interdisziplinäre Synthese: Autorschaft im Zeitalter generativer KI}
Die vorliegende Arbeit hat Autorschaft im Kontext generativer Künstlicher Intelligenz aus
philosophischer, technischer, ethischer und rechtlicher Perspektive untersucht. In der
interdisziplinären Synthese werden diese Perspektiven nicht lediglich addiert, sondern in
ihrem Zusammenspiel reflektiert. Ziel ist es, ein übergreifendes Verständnis von Autorschaft
zu entwickeln, das der veränderten Werkgenese durch KI-Systeme gerecht wird, ohne zentrale
normative Ansprüche preiszugeben.

\subsection{Autorschaft zwischen Subjekt, Prozess und Zuschreibung}
Die philosophischen Grundlagen haben gezeigt, dass Autorschaft historisch eng mit dem
schöpferischen Subjekt verbunden ist. In der aufklärerischen und ästhetischen Tradition,
insbesondere bei Kant, wird der Autor als intentional handelnde Person verstanden, deren
Werk Ausdruck individueller Geistestätigkeit ist. Autorschaft ist hier an Originalität,
Verantwortung und normative Zurechenbarkeit gebunden.

Die technische Analyse generativer KI stellt dieses Verständnis grundlegend infrage.
KI-generierte Werke entstehen nicht aus bewusster Intentionalität, sondern aus statistischen,
daten- und regelbasierten Prozessen. Die Werkgenese ist verteilt: Trainingsdaten,
Modellarchitekturen, Entwicklerentscheidungen und Nutzereingaben tragen jeweils in
unterschiedlicher Weise zum Ergebnis bei. Autorschaft erscheint hier weniger als Eigenschaft
eines Subjekts denn als Beschreibung eines komplexen soziotechnischen Prozesses.

Die Synthese dieser Perspektiven macht deutlich, dass es sich nicht um einen einfachen
Widerspruch handelt. Während die Technik beschreibt, wie Werke entstehen, fragt die
Philosophie danach, wem Autorschaft sinnvoll zugeschrieben werden kann. Autorschaft bewegt
sich damit zwischen subjektiver Zuschreibung, prozessualer Beschreibung und normativer
Funktion.

\subsection{Verteilte Kreativität und normative Leerstellen}
Die in den Grundlagen entwickelte Unterscheidung zwischen Kreativität und Autorschaft erweist
sich in der interdisziplinären Betrachtung als zentral. Wie gezeigt wurde, können
KI-Systeme funktionale Kreativität realisieren, ohne die Voraussetzungen klassischer
Autorschaft zu erfüllen. Kreative Neuheit allein genügt daher nicht, um Autorschaft zu
begründen.

Diese Einsicht erzeugt normative Leerstellen. Aus der deskriptiven Feststellung verteilter
kreativer Beiträge folgt noch keine Antwort auf Fragen der Verantwortung, Anerkennung oder
Fairness. Gerade weil KI kreativ erscheinen kann, ohne selbst Autor zu sein, verschiebt sich
der normative Fokus von der Zuschreibung kreativer Leistung hin zur Gestaltung
verantwortungsvoller Praktiken im Umgang mit solchen Systemen.

Hier setzt die ethische Perspektive an. Autorschaft wird nicht primär als Eigentumsanspruch
verstanden, sondern als Bündel moralischer Beziehungen. Zentrale Anforderungen betreffen
Transparenz, Anerkennung menschlicher Vorleistungen und faire Verfahren in einem
kreativen Informationsökosystem. Besonders problematisch ist die strukturelle Unsichtbarkeit
derjenigen Beiträge, die als Trainingsdaten genutzt werden, ohne dass ihre Urheberinnen und
Urheber beteiligt oder anerkannt werden.

Die Black-Box-Problematik ist damit nicht nur ein technisches, sondern ein normatives Problem:
Wenn Herkunft, Einfluss und Nähe zu bestehenden Werken nicht nachvollziehbar sind, wird
Anerkennung systematisch erschwert. Autorschaft fungiert hier weniger als Zuschreibung von
Kreativität, sondern als Orientierungsbegriff für verantwortungsvolle Praxis.

\subsection{Rechtliche Fixierung und ihre Grenzen}
Die rechtliche Perspektive stabilisiert Autorschaft normativ, indem sie sie an die menschliche
Person bindet. Das geltende Urheberrecht erkennt nur natürliche Personen als Urheber an und
verneint Autorschaft bei rein KI-generierten Werken. Diese Fixierung schafft Rechtssicherheit,
steht jedoch in einem Spannungsverhältnis zur faktischen Werkgenese.

In der interdisziplinären Zusammenschau wird deutlich, dass das Recht bewusst vereinfacht, wo
Technik und Ethik komplexe Verhältnisse beschreiben. Diese Vereinfachung ist funktional
notwendig, führt jedoch zu Friktionen: Moralische Ansprüche auf Anerkennung oder
Kompensation lassen sich nicht immer rechtlich abbilden.

Der AI Act reagiert auf diese Spannung weniger durch eine Neudefinition von Autorschaft als
durch Transparenz-, Dokumentations- und Governance-Pflichten. Autorschaft fungiert rechtlich
damit weniger als Beschreibung kreativer Realität, sondern als normativer Ankerpunkt zur
Ordnung von Verantwortung, Haftung und Verwertung.
\subsection{Autorschaft als relationale Kategorie}
Aus der Zusammenführung aller Perspektiven ergibt sich ein zentrales Ergebnis: Autorschaft im
Zeitalter generativer KI lässt sich weder sinnvoll auf das menschliche Subjekt allein noch auf
das technische System übertragen. Stattdessen erweist sie sich als relationale Kategorie, die
unterschiedliche Rollen, Beiträge und Verantwortlichkeiten integriert.

Technisch ist Autorschaft verteilt, ethisch ist sie verantwortungsgebunden, rechtlich ist sie
personenzentriert. Diese Differenzen sind kein Defizit, sondern Ausdruck unterschiedlicher
normativer Funktionen. Autorschaft fungiert damit weniger als eindeutige Zuschreibung denn
als Schnittstelle zwischen Beschreibung, Bewertung und Regulierung.

Die interdisziplinäre Synthese legt nahe, Autorschaft nicht aufzugeben, sondern neu zu
kontextualisieren: nicht als exklusives Eigentum, sondern als Orientierungsbegriff für
Transparenz, Anerkennung und faire Praxis in zunehmend hybriden kreativen Prozessen.