\section{Rechtliche Perspektive: Schutz geistiger Leistung im KI-Kontext}

Die rechtliche Bewertung von KI-generierten Inhalten stellt den Gesetzgeber und die Rechtsprechung vor neue Herausforderungen. Während das Urheberrecht traditionell auf den menschlichen Schöpfer ausgerichtet ist, stellen generative KI-Systeme diese Sichtweise infrage. Dieses Kapitel beleuchtet die aktuellen urheberrechtlichen Grundlagen, die Eigentumsfrage an KI-Werken sowie die Auswirkungen der EU-KI-Verordnung.

\subsection{Aktuelle urheberrechtliche Grundlagen}
Das deutsche Urheberrecht schützt nach § 2 Abs. 2 UrhG \enquote{ nurpersönliche geistige Schöpfungen}. Dieser Werkbegriff setzt zwingend ein menschliches Handeln voraus. Der Urheber ist gemäß § 7 UrhG \enquote{der Schöpfer des Werkes}, also eine natürliche Person. Eine Maschine oder ein Algorithmus kann nach geltendem Recht nicht als Urheber fungieren, da ihnen die notwendige Rechtsfähigkeit fehlt. KI-Systeme sind juristisch betrachtet Objekte und keine Rechtssubjekte. Somit können sie keine Willenserklärungen abgeben und somit auch keine Rechte an einem Werk erwerben.

Ein zentrales Kriterium für den Urheberrechtsschutz ist die sogenannte Schöpfungshöhe. Das Werk muss eine gewisse Individualität aufweisen, die sich von alltäglichen, rein handwerklichen oder automatisierten Erzeugnissen abhebt. Bei KI-generierten Inhalten fehlt es oft an dieser unmittelbaren menschlichen Gestaltungskraft im Endergebnis, da der Generierungsprozess maßgeblich durch den Algorithmus gesteuert wird.

\subsection{Eigentum an KI-generierten Werken}
Die Frage, wem die Rechte an einem mittels KI erzeugten Bild oder Text zustehen, ist komplex. Grundsätzlich gilt: Wenn ein KI-System ohne einen nennenswerten menschlichen Beitrag einen Inhalt erstellt, ist dieser nach aktuellem Stand in Deutschland und der EU gemeinfrei. Es entsteht also kein Urheberrechtsschutz.

Ein prominentes Fallbeispiel ist das Bild \enquote{Théâtre D’opéra Spatial} von Jason Allen, das 2022 einen Kunstwettbewerb gewann. Das US Copyright Office verweigerte dem Bild den Urheberrechtsschutz mit der Begründung, dass es an der erforderlichen menschlichen Urheberschaft fehle \cite[vgl.][]{CopyrightGov}. Zwar hatte Allen umfangreiche Prompts  verwendet und das Bild nachbearbeitet, doch der wesentliche bildgebende Prozess wurde der KI ,Midjourney, zugeschrieben. Diese Entscheidung spiegelt auch die tendenzielle Rechtsauffassung in Europa wider: Der bloße Auftrag, in diesem Falle das Prompting, an eine KI reicht in der Regel nicht aus, um als Urheber des Ergebnisses zu gelten, es sei denn, der Prompt selbst erreicht Schöpfungshöhe (z. B. als Sprachwerk) oder die menschliche Nachbearbeitung ist so dominant, dass sie dem Werk das Gepräge gibt.

In der Praxis bedeutet dies, dass reine KI-Erzeugnisse oft von jedermann genutzt werden können. Dies wirft wirtschaftliche Fragen für Kreativschaffende auf, die KI als Werkzeug nutzen, aber möglicherweise keine exklusiven Verwertungsrechte an den Roh-Ergebnissen erhalten. 

\subsection{Perspektive der EU-KI-Verordnung}
Mit dem \enquote{AI Act} (KI-Verordnung) hat die Europäische Union einen umfassenden Rechtsrahmen geschaffen, der auch Auswirkungen auf das Urheberrecht hat. Zwar regelt der AI Act nicht primär das Eigentum an Werken, er führt jedoch wichtige Transparenz- und Kennzeichnungspflichten ein.

Gemäß der Verordnung müssen Anbieter von KI-Systemen sicherstellen, dass KI-generierte Inhalte als solche erkennbar sind. Dies gilt insbesondere für Deepfakes oder Texte, die über Bots verbreitet werden. Diese Kennzeichnungspflicht dient dem Verbraucherschutz und der Desinformationsbekämpfung, hat aber auch eine urheberrechtliche Dimension: Sie erleichtert die Unterscheidung zwischen menschlichen und maschinellen Leistungen. Zudem verpflichtet der AI Act Anbieter von General-Purpose AI Models (GPAI), detaillierte Zusammenfassungen der für das Training verwendeten Inhalte zu veröffentlichen. Dies stärkt die Position von Urhebern, deren Werke zum Training von KI genutzt wurden, und ermöglicht potenziell die Durchsetzung von Vergütungsansprüchen \cite[vgl.][]{eu_ai_act_2024}.

Für die Autorschaftsdefinition bedeutet dies, dass die Grenze zwischen Mensch und Maschine rechtlich schärfer gezogen werden soll. Die Transparenzpflichten zwingen dazu, den Anteil der KI an einem Werk offenzulegen, was die unbemerkte Aneignung maschineller Leistungen als eigene menschliche Schöpfung erschweren soll.

\subsection{Zwischenfazit: Rechtliche Neudefinition geistiger Schöpfung}
Die rechtliche Analyse zeigt, dass das bestehende Urheberrechtssystem unter Druck gerät. Der Grundsatz, dass nur Menschen Urheber sein können, bleibt zwar bestehen, führt aber bei der zunehmenden Qualität und Verbreitung von KI-Werken zu Regelungslücken und einer Ausweitung gemeinfreier Inhalte. Die EU-KI-Verordnung setzt hier mit Transparenzpflichten an, löst aber nicht die eigentumsrechtliche Kernfrage. Es zeichnet sich ab, dass der Begriff der \enquote{geistigen Schöpfung} in Zukunft möglicherweise differenzierter betrachtet werden muss, um das Zusammenspiel von menschlicher Intention und maschineller Ausführung adäquat abzubilden.
