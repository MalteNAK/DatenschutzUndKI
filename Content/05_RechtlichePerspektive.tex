\section{Rechtliche Perspektive: Schutz geistiger Leistung und Daten im KI-Kontext}

Die rechtliche Bewertung von KI-generierten Inhalten ist für diese Arbeit vor allem dort relevant, wo sie in technische Anforderungen übersetzbar wird. Aus Sicht der Informatik sind Rechtstexte weniger \enquote{nur} Restriktionen, sondern beschreiben Randbedingungen für Systemdesign, Datenflüsse, Dokumentation und Verantwortlichkeiten. Dieses Kapitel skizziert daher die urheberrechtliche Grundlinie zur menschlichen Urheberschaft und Implikationen des AI Act für Transparenz und Governance\cite[vgl.][]{eu_ai_act_2024}.

\subsection{Urheberrechtliche Grundlinie: menschliche Urheberschaft und Nachweisbarkeit}
Das deutsche Urheberrecht schützt nach § 2 Abs. 2 UrhG \enquote{nur persönliche geistige Schöpfungen}. Der Urheber ist gemäß § 7 UrhG \enquote{der Schöpfer des Werkes}, also eine natürliche Person. Ein KI-System kann daher nicht selbst Urheber sein.

Für KI-gestützte Kreativität ist technisch vor allem die Konsequenz entscheidend: Urheberrechtlicher Schutz folgt nicht automatisch aus der Nutzung eines Tools, sondern hängt davon ab, ob ein individueller, gestaltender menschlicher Beitrag am konkreten Ergebnis nachvollziehbar ist (z.\,B. durch Auswahl-, Bearbeitungs- oder Kompositionsentscheidungen).

Ein prominentes Fallbeispiel ist das Bild \enquote{Théâtre D’opéra Spatial} von Jason Allen, bei dem das US Copyright Office Urheberrechtsschutz verneinte, weil es an hinreichender menschlicher Urheberschaft fehle\cite[vgl.][]{CopyrightGov}. Auch wenn die Rechtslage in den USA nicht unmittelbar auf Deutschland übertragbar ist, illustriert der Fall den allgemeinen Trend: Je autonomer ein System das Ergebnis prägt, desto schwieriger wird die Zuschreibung.

Für technische Kontexte folgt daraus: Wenn Organisationen KI kreativ einsetzen, werden Prozessartefakte (Prompt-Versionen, Iterationen, Auswahlentscheidungen, Nachbearbeitungsschritte, Modellversion) zu wichtigen Nachweisen, um menschliche Beiträge plausibel zu dokumentieren. Abzuwarten bleibt, wie Gerichte in Deutschland bzw. innerhalb der EU solche Nachweise künftig bewerten. Klar ist jedoch, dass Promopts ohne weitere Bearbeitung oder Auswahlentscheidungen in der Regel nicht als hinreichende menschliche Urheberschaft gelten dürften.

\subsection{EU-KI-Verordnung (AI Act): Transparenz, Dokumentation und Governance}
Der AI Act ergänzt die Urheberrechtsfrage um einen risikobasierten Regulierungsrahmen für KI-Systeme. Für generative KI und General-Purpose AI Models (GPAI) sind insbesondere Pflichten relevant, die sich in technische Umsetzungsaufgaben übersetzen lassen\cite[vgl.][]{eu_ai_act_2024}.

Aus technischer Perspektive sind vier Umsetzungsfelder zentral:
\begin{enumerate}
	\item \textbf{Kennzeichnung und Herkunft:} Mechanismen, die KI-generierte Inhalte als solche erkennbar machen (z.\,B. UI-Hinweise, Metadaten, Watermarking-Ansätze) – besonders in Kontexten mit Täuschungs- oder Desinformationsrisiken.
	\item \textbf{Dokumentation und Nachweisfähigkeit:} Versionierung von Modellen, Datenpipelines und Evaluationsartefakten sowie technische Dokumentation (z.\,B. Model Cards/Datasheets) unterstützen die Nachvollziehbarkeit von Fähigkeiten, Grenzen und Updates.
	\item \textbf{Transparenz über Trainingsdaten:} Zusammenfassungen zur Art der Trainingsdaten und Datenquellen helfen Governance, Audits und die Bewertung von Risiken (z.\,B. Bias, Datenherkunft, urheberrechtliche Konflikte).
	\item \textbf{Betriebliches Risikomanagement:} Monitoring, Incident-Handling und klare Verantwortlichkeiten (Anbietende/Betreibende/Nutzende) werden Teil des Systems, nicht nur des \enquote{Drumherums}.
\end{enumerate}

Für die Autorschaftsdebatte bedeutet das weniger eine rechtliche \enquote{Neudefinition des Autors} als eine stärkere Verpflichtung zur Transparenz über maschinelle Anteile. Technisch gewinnen damit Provenienz- und Prozessinformationen an Bedeutung (z.\,B. Modellversion, Erzeugungsmodus, Bearbeitungsschritte), um Verantwortung und Integrität nachvollziehbar zu halten.
Weiterhin unklar bleibt jedoch, welche Quellen genau im Einzelfall zur Generierung eines Outputs beigetragen haben – eine Herausforderung, die technische Transparenzmethoden (XAI) nur teilweise adressieren können. Dadurch bestetht weiterhin die Gefahr struktureller Anerkennungsdefizite für kreative Vorleistungen, die in Trainingsdaten genutzt werden.

\subsection{Zwischenfazit: Rechtliche Neudefinition geistiger Schöpfung}
Die rechtliche Betrachtung zeigt, dass das Urheberrecht an einem personenzentrierten Autorschaftsverständnis festhält, auch wenn die faktische Werkgenese zunehmend technisch vermittelt ist. Diese normative Fixierung dient der Rechtssicherheit, kann jedoch die Komplexität verteilter kreativer Prozesse nur eingeschränkt abbilden.

Aus rechtlicher Perspektive wird deutlich, dass Gesetze nicht primär beschreiben, wie kreative Leistungen entstehen, sondern Ordnung schaffen, indem sie Verantwortung, Schutz und Verwertungsrechte eindeutig zuweisen. Dabei entstehen Spannungen zu technischen und ethischen Perspektiven, die Autorschaft relationaler und gradueller denken. Auch sind diese Gesetzte tehnologieoffen gestaltendet, sodass sie zwar Rahmenbedingungen setzen, aber dem technischen Wandel nicht im Wege stehen.

Autorschaft fungiert rechtlich somit weniger als Abbild kreativer Realität und mehr als stabilisierende Kategorie. Die verbleibenden Differenzen zu technischen Beschreibungen und ethischen Bewertungen markieren keine Fehlstellen einzelner Disziplinen, sondern verweisen auf die Notwendigkeit einer interdisziplinären Gesamtsicht, wie sie im abschließenden Synthesekapitel vorgenommen wird.
